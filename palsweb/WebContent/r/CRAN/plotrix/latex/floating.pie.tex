\HeaderA{floating.pie}{Display a floating pie chart}{floating.pie}
\keyword{misc}{floating.pie}
\begin{Description}\relax
Displays a pie chart at an arbitrary position on an existing plot
\end{Description}
\begin{Usage}
\begin{verbatim}
 floating.pie(xpos,ypos,x,edges=200,radius=1,col=NULL,startpos=0,
  shadow=FALSE,...)
\end{verbatim}
\end{Usage}
\begin{Arguments}
\begin{ldescription}
\item[\code{xpos,ypos}] x and y position of the center of the pie chart
\item[\code{x}] a numeric vector for which each value will be a sector
\item[\code{edges}] the number of lines forming a circle
\item[\code{radius}] the radius of the pie in user units
\item[\code{col}] the colors of the sectors - defaults to \samp{rainbow}
\item[\code{startpos}] The starting position for drawing sectors in radians.
\item[\code{shadow}] Logical - whether to draw a shadow
\item[\code{...}] graphical parameters passed to \samp{polygon}
\end{ldescription}
\end{Arguments}
\begin{Value}
The bisecting angle of the sectors in radians. Useful for placing
text labels for each sector.
\end{Value}
\begin{Note}\relax
As with most pie charts, simplicity is essential. Trying to display a
complicated breakdown of data rarely succeeds.
\end{Note}
\begin{Author}\relax
Jim Lemon
\end{Author}
\begin{SeeAlso}\relax
\samp{\LinkA{pie.labels}{pie.labels}}, \samp{\LinkA{boxed.labels}{boxed.labels}},
\samp{\LinkA{polygon.shadow}{polygon.shadow}}
\end{SeeAlso}
\begin{Examples}
\begin{ExampleCode}
 plot(1:5,type="n",main="Floating Pie test",xlab="",ylab="",axes=FALSE)
 box()
 polygon(c(0,0,5.5,5.5),c(0,3,3,0),border="#44aaff",col="#44aaff")
 floating.pie(1.7,3,c(2,4,4,2,8),radius=0.5,
  col=c("#ff0000","#80ff00","#00ffff","#44bbff","#8000ff"))
 floating.pie(3.1,3,c(1,4,5,2,8),radius=0.5,
  col=c("#ff0000","#80ff00","#00ffff","#44bbff","#8000ff"))
 floating.pie(4,1.5,c(3,4,6,7),radius=0.5,
  col=c("#ff0066","#00cc88","#44bbff","#8000ff"))
 draw.circle(3.9,2.1,radius=0.04,col="white")
 draw.circle(3.9,2.1,radius=0.04,col="white")
 draw.circle(3.9,2.1,radius=0.04,col="white")
 draw.circle(4,2.3,radius=0.04,col="white")
 draw.circle(4.07,2.55,radius=0.04,col="white")
 draw.circle(4.03,2.85,radius=0.04,col="white")
 text(c(1.7,3.1,4),c(3.7,3.7,3.7),c("Pass","Pass","Fail"))
\end{ExampleCode}
\end{Examples}

