\HeaderA{getIntersectList}{Enter the information for a set intersection display}{getIntersectList}
\keyword{misc}{getIntersectList}
\begin{Description}\relax
Enter the information for a set intersection display.
\end{Description}
\begin{Usage}
\begin{verbatim}
 getIntersectList(nelem,xnames=NULL)
\end{verbatim}
\end{Usage}
\begin{Arguments}
\begin{ldescription}
\item[\code{nelem}] The number of sets for which the intersections will be displayed.
\item[\code{xnames}] The labels for the set intersections. The function creates names
from combinations of the first \samp{nelem} capital letters if none are 
given.
\end{ldescription}
\end{Arguments}
\begin{Details}\relax
\samp{getIntersectList} allows the user to manually enter the counts of
set intersections rather than build this information from a matrix of data.
It is probably most useful for producing an intersection diagram when the
counts of the intersections are already known.
\end{Details}
\begin{Value}
A list of the counts of elements in the set intersections.
\end{Value}
\begin{Author}\relax
Jim Lemon
\end{Author}
\begin{SeeAlso}\relax
\samp{\LinkA{makeIntersectList}{makeIntersectList}}, \samp{\LinkA{intersectDiagram}{intersectDiagram}}
\end{SeeAlso}

