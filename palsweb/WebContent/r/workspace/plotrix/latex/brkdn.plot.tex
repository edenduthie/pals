\HeaderA{brkdn.plot}{A point/line plotting routine}{brkdn.plot}
\keyword{misc}{brkdn.plot}
\begin{Description}\relax
Display a point/line plot of breakdowns of one or more variables.
\end{Description}
\begin{Usage}
\begin{verbatim}
 brkdn.plot(vars,groups=NA,obs=NA,data,mct="mean",md="std.error",stagger=NA,
 dispbar=TRUE,main="Breakdown plot",xlab=NA,ylab=NA,xaxlab=NA,
 ylim=NA,type="b",pch=1,lty=1,col=par("fg"),staxx=FALSE,...)
\end{verbatim}
\end{Usage}
\begin{Arguments}
\begin{ldescription}
\item[\code{vars}] The names or indices of one or more columns in a data frame.
The columns must contain numeric data.
\item[\code{groups}] The name or index of a column in a data frame that classifies
the values in \samp{vars} into different, usually fixed effect, levels.
\item[\code{obs}] The name or index of a column in a data frame that classifies
the values in \samp{vars} into different, usually random effect, levels.
\item[\code{data}] The data frame.
\item[\code{mct}] The measure of central tendency to calculate for each group.
\item[\code{md}] The measure of dispersion to calculate, NA for none.
\item[\code{stagger}] The amount to offset the successive values at each horizontal
position as a proportion of the width of the plot. The calculated default
is usually adequate. Pass zero for none.
\item[\code{dispbar}] Whether to display the measures of dispersion as bars.
\item[\code{main}] The title at the top of the plot.
\item[\code{xlab,ylab}] The labels for the X and Y axes respectively. There are
defaults, but they are basic.
\item[\code{xaxlab}] Optional labels for the horizontal axis ticks.
\item[\code{ylim}] Optional vertical limits for the plot.
\item[\code{type}] Whether to plot symbols, lines or both (as in \samp{plot}).
\item[\code{pch}] Symbol(s) to plot.
\item[\code{lty}] Line type(s) to plot.
\item[\code{col}] Color(s) for the symbols and lines.
\item[\code{staxx}] Whether to call \samp{\LinkA{staxlab}{staxlab}} to display the X axis
labels.
\item[\code{...}] additional arguments passed to \samp{plot}.
\end{ldescription}
\end{Arguments}
\begin{Details}\relax
\samp{brkdn.plot} displays a plot useful for visualizing the breakdown of a
response measure by two factors, or more than one response measure by either
a factor representing something like levels of treatment (\samp{groups}) or
something like repeated observations (\samp{obs}). For example, if
observations are made at different times on data objects that receive
different treatments, the \samp{groups} factor will display the measures
of central tendency as points/lines with the same color, symbol and line type,
while the \samp{obs} factor will be represented as horizontal positions on the
plot. If \samp{obs} is numeric, its unique values will be used as the 
positions, if not, 1 to the number of unique values. This is a common way of 
representing changes over time intervals for experimental groups.
\end{Details}
\begin{Value}
A list of two matrices of dimension \samp{length(levels(groups))} by 
\samp{length(levels(obs))}. The first contains the measures of central
tendency calculated and its name is the name of the function passed as
\samp{mct}. The second contains the measures of dispersion and its name
is the name of the function passed as \samp{md}.

If both \samp{groups} and \samp{obs} are not NA, the rows of each matrix
will be the \samp{groups} and the columns the \samp{obs}. If \samp{obs}
is NA, the rows will be the \samp{groups} and the columns the \samp{vars}.
If \samp{groups} is NA, the rows will be the \samp{vars} and the columns
the \samp{obs}. That is, if \samp{vars} has more than one element, if
\samp{obs} is NA, the elements of \samp{vars} will be considered to
represent observations, while if \samp{groups} is NA, they will be
considered to represent groups. At least one of \samp{groups} and \samp{obs}
must be not NA or there is no point in using \samp{brkdn.plot}.
\end{Value}
\begin{Author}\relax
Jim Lemon
\end{Author}
\begin{SeeAlso}\relax
\samp{\LinkA{dispbars}{dispbars}}
\end{SeeAlso}
\begin{Examples}
\begin{ExampleCode}
 test.df<-data.frame(a=rnorm(80)+4,b=rnorm(80)+4,c=rep(LETTERS[1:4],each=20),
  d=rep(rep(letters[1:4],each=4),5))
 # first use the default values
 brkdn.plot("a","c","d",test.df,pch=1:4,col=1:4)
 # now jazz it up a bit using medians and median absolute deviations
 # and some enhancements
 bp<-brkdn.plot("a","c","d",test.df,main="Test of the breakdown plot",
  mct="median",md="mad",xlab="Temperature range", ylab="Cognition",
  xaxlab=c("10-15","16-20","21-25","25-30"),pch=1:4,lty=1:4,col=1:4)
 es<-emptyspace(bp)
 legend(es,legend=c("Sydney","Gosford","Karuah","Brisbane"),pch=1:4,
  col=1:4,lty=1:4,xjust=0.5,yjust=0.5)
\end{ExampleCode}
\end{Examples}

