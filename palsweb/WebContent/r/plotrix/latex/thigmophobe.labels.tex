\HeaderA{thigmophobe.labels}{Place labels away from the nearest point}{thigmophobe.labels}
\keyword{misc}{thigmophobe.labels}
\begin{Description}\relax
\samp{thigmophobe.labels} places labels adjacent to each point, 
offsetting each label in the direction returned by 
\samp{thigmophobe}.
\end{Description}
\begin{Usage}
\begin{verbatim}
 thigmophobe.labels(x,y,labels=NULL,text.pos=NULL,...)
\end{verbatim}
\end{Usage}
\begin{Arguments}
\begin{ldescription}
\item[\code{x,y}] Numeric data vectors or a list with two components. 
Typically the x/y coordinates of plotted points.
\item[\code{labels}] A vector of strings that will be placed adjacent to
each point. Defaults to the indices of the coordinates.
\item[\code{text.pos}] An optional vector of text positions (see 
\samp{\LinkA{text}{text}}).
\item[\code{...}] additional arguments are passed to \samp{text}
\end{ldescription}
.
\end{Arguments}
\begin{Details}\relax
Typically used to automatically place labels on a scatterplot or 
similar to avoid overlapping labels. \samp{thigmophobe.labels}
will sometimes place a label off the plot or fail to separate
labels in clusters of points. The user can manually adjust the
errant labels by running \samp{thigmophobe} first and saving
the returned vector. Then modify the position values to place
the labels properly and pass the edited vector to
\samp{thigmophobe.labels} as the \samp{text.pos} argument. This
takes precedence over the positions calculated by \samp{thigmophobe}.

Both \samp{pointLabel} in the \pkg{maptools} package and \samp{spread.labs}
in the \pkg{TeachingDemos} package use more sophisticated algorithms to
place the labels and are worth a try if \samp{thigmophobe} just won't get it
right.
\end{Details}
\begin{Value}
A vector of directions away from the point nearest to each point.
\end{Value}
\begin{Author}\relax
Jim Lemon
\end{Author}
\begin{SeeAlso}\relax
\samp{\LinkA{thigmophobe}{thigmophobe}}, \samp{\LinkA{text}{text}}
\end{SeeAlso}
\begin{Examples}
\begin{ExampleCode}
 x<-rnorm(20)
 y<-rnorm(20)
 xlim<-range(x)
 xspace<-(xlim[2]-xlim[1])/20
 xlim<-c(xlim[1]-xspace,xlim[2]+xspace)
 ylim<-range(y)
 yspace<-(ylim[2]-ylim[1])/20
 ylim<-c(ylim[1]-yspace,ylim[2]+yspace)
 plotlabels<-
  c("one","two","three","four","five","six","seven","eight","nine","ten",
  "eleven","twelve","thirteen","fourteen","fifteen","sixteen","seventeen",
  "eighteen","nineteen","twenty")
 plot(x=x,y=y,xlim=xlim,ylim=ylim,main="Test thigmophobe.labels")
 # skip the almost invisible yellow label, make them bold and without borders
 thigmophobe.labels(x,y,plotlabels,col=c(2:6,8:12),border=NA,font=2)
\end{ExampleCode}
\end{Examples}

