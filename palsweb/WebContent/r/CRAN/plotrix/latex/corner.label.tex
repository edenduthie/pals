\HeaderA{corner.label}{Find corner locations and optionally display a label}{corner.label}
\keyword{aplot}{corner.label}
\begin{Description}\relax
Finds the coordinates in user parameters of a specified corner of the
figure region and optionally displays a label there
\end{Description}
\begin{Usage}
\begin{verbatim}
 corner.label(label=NULL,x=-1,y=1,xoff=NA,yoff=NA,figcorner=FALSE,...)
\end{verbatim}
\end{Usage}
\begin{Arguments}
\begin{ldescription}
\item[\code{label}] Text to display. The default is to display nothing.
\item[\code{x}] an integer value: -1 for the left side of the plot, 1 for the
right side
\item[\code{y}] an integer value: -1 for the bottom side of the plot, 1 for
the top side
\item[\code{xoff,yoff}] Horizontal and vertical text offsets. Defaults to one
half of the width and height of "m" respectively. 
\item[\code{figcorner}] Whether to find/display at the corner of the plot or figure.
\item[\code{...}] further arguments to the \samp{text} command for the label
\end{ldescription}
\end{Arguments}
\begin{Details}\relax
\samp{corner.label} finds the specified corner of the plot or figure and if
\samp{label} is not NULL, displays it there. The text justification is specified
so that the label will be justified away from the corner. To get the label
squeezed right into a corner, set \samp{xoff} and \samp{yoff} to zero.
\end{Details}
\begin{Value}
A list of the x and y positions of the corner adjusted for the offsets.
\end{Value}
\begin{Author}\relax
Ben Bolker
\end{Author}
\begin{Examples}
\begin{ExampleCode}
 plot(1:10,1:10)
 corner.label("A")
 corner.label(x=1,y=1)
 corner.label("B",y=-1,x=1,figcorner=TRUE,col="red")
\end{ExampleCode}
\end{Examples}

