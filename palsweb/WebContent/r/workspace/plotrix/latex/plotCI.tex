\HeaderA{plotCI}{Plot confidence intervals/error bars}{plotCI}
\keyword{hplot}{plotCI}
\begin{Description}\relax
Given a set of x and y values and upper and lower bounds,
this function plots the points with error bars.
\end{Description}
\begin{Usage}
\begin{verbatim}
 plotCI(x,y=NULL,uiw,liw=uiw,ui=NULL,li=NULL,err="y",
  sfrac=0.01,gap=0,slty=par("lty"),add=FALSE,scol=NULL,pt.bg=par("bg"),...)
\end{verbatim}
\end{Usage}
\begin{Arguments}
\begin{ldescription}
\item[\code{x}] The x coordinates of points in the plot
\item[\code{y}] The y coordinates of points in the plot
\item[\code{uiw}] The width of the upper portion of the confidence region,
or (if \samp{liw} is missing) the width of both halves of
the confidence region
\item[\code{liw}] The width of the lower portion of the confidence region (if
missing, the function assumes symmetric confidence bounds)
\item[\code{ui}] The absolute upper limit of the confidence region
\item[\code{li}] The absolute lower limit of the confidence region
\item[\code{err}] The direction of error bars: "x" for horizontal, "y"
for vertical ("xy" would be nice but is not implemented yet; don't
know quite how everything would be specified.  See examples for
composing a plot with simultaneous horizontal and vertical error bars)
\item[\code{gap}] Size of gap in error bars around points
(default 0;gap=TRUE gives gap size of 0.01)
\item[\code{sfrac}] Scaling factor for the size of the "serifs" (end bars) on
the confidence bars, in x-axis units
\item[\code{add}] If FALSE (default), create a new plot; if TRUE, add error
bars to an existing plot.
\item[\code{slty}] Line type of error bars
\item[\code{scol}] Color of error bars: if \samp{col} is specified in the
optional arguments, \samp{scol} is set the same; otherwise it's set
to \samp{par(scol)}
\item[\code{pt.bg}] Background color of points (use pch=21, pt.bg=par("bg")
to get open points superimposed on error bars)
\item[\code{...}] Any other parameters to be passed through to
\samp{\LinkA{plot.default}{plot.default}}, \samp{\LinkA{points}{points}},
\samp{\LinkA{arrows}{arrows}}, etc. (e.g. \samp{lwd}, \samp{col}, \samp{pch},
\samp{axes}, \samp{xlim}, \samp{ylim}).  \samp{xlim} and
\samp{ylim} are set by default to include all of the data points and
error bars.  \samp{xlab} and \samp{ylab} are set to the names of
\samp{x} and \samp{y}.  If \samp{pch==NA}, no points are drawn
(e.g. leaving room for text labels instead)
\end{ldescription}
\end{Arguments}
\begin{Value}
invisible(x,y); creates a plot on the current device.
\end{Value}
\begin{Author}\relax
Ben Bolker (documentation and tweaking of a function provided by
Bill Venables, additional feature ideas from Gregory Warnes)
\end{Author}
\begin{SeeAlso}\relax
\samp{\LinkA{boxplot}{boxplot}}
\end{SeeAlso}
\begin{Examples}
\begin{ExampleCode}
 y<-runif(10)
 err<-runif(10)
 plotCI(1:10,y,err)
 plotCI(1:10,y,err,2*err,lwd=2,col="red",scol="blue")
 err.x<-runif(10)
 err.y<-runif(10)
 plotCI(1:10,y,err.y,pt.bg=par("bg"),pch=21)
 plotCI(1:10,y,err.x,pt.bg=par("bg"),pch=21,err="x",add=TRUE)
 data(warpbreaks)
 attach(warpbreaks)
 wmeans<-by(breaks,tension,mean)
 wsd<-by(breaks,tension,sd)
 ## note that barplot() returns the midpoints of the bars, which plotCI
 ##  uses as x-coordinates
 plotCI(barplot(wmeans,col="gray",ylim=c(0,max(wmeans+wsd))),wmeans,wsd,add=TRUE)
 ## using labels instead of points
 labs<-sample(LETTERS,replace=TRUE,size=10)
 plotCI(1:10,y,err,pch=NA,gap=0.02)
 text(1:10,y,labs)
\end{ExampleCode}
\end{Examples}

