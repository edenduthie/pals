\HeaderA{pyramid.plot}{Pyramid plot}{pyramid.plot}
\keyword{misc}{pyramid.plot}
\begin{Description}\relax
Displays a pyramid (opposed horizontal bar) plot on the current
graphics device.
\end{Description}
\begin{Usage}
\begin{verbatim}
 pyramid.plot(xy,xx,labels,top.labels=c("Male","Age","Female"),
  main="",laxlab=NULL,raxlab=NULL,unit="%",xycol,xxcol,gap=1,
  labelcex=1,mark.cat=NA,add=FALSE)
\end{verbatim}
\end{Usage}
\begin{Arguments}
\begin{ldescription}
\item[\code{xy,xx}] Vectors of percentages (but see Details) both of which
should total 100 if this is a population pyramid, and be of equal length.
\item[\code{labels}] Labels for the categories represented by each pair of
bars. There should be a label for each xy or xx value, even if empty.
\item[\code{top.labels}] The two categories represented on the left and right
sides of the plot and a heading for the labels up the center.
\item[\code{main}] Optional title for the plot.
\item[\code{laxlab}] Optional labels for the left x axis ticks.
\item[\code{raxlab}] Optional labels for the right x axis ticks.
\item[\code{unit}] The label for the units of the plot.
\item[\code{xycol,xxcol}] Color(s) for the left and right sets of bars. Both of
these default to 1:length(labels).
\item[\code{gap}] One half of the space between the two sets of bars for the
\samp{labels} in user units.
\item[\code{labelcex}] Expansion for the category labels.
\item[\code{mark.cat}] If an integer equal to the index of one of the labels
is passed, that label will have a rectangle drawn around it.
\item[\code{add}] Whether to add bars to an existing plot. Usually this
involves overplotting a second set of bars, perhaps transparent.
\end{ldescription}
\end{Arguments}
\begin{Details}\relax
\samp{pyramid.plot} is principally intended for population pyramids,
although it can display other types of opposed bar charts with suitable
modification of the arguments. If the user wants a different unit for
the display, just change \samp{unit} accordingly. The default gap of 
two units is usually satisfactory for the four to six percent range 
of most bars on population pyramids.

If \samp{xy} and \samp{xx} are matrices or data frames, 
\samp{pyramid.plot} will produce opposed stacked bars with the first 
columns innermost. In this mode, colors are limited to one per column. The 
stacked bar mode will in general not work with the \samp{add} method. Note 
that the stacked bar mode can get very messy very quickly.

The \samp{add} argument allows one or more sets of bars to be plotted
on an existing plot. If these are not transparent, any bar that is
shorter than the bar that overplots it will disappear. Only some graphic
devices (e.g. \samp{pdf}) will handle transparency.

In order to add bars, the function cannot restore the initial margin values
or the new bars will not plot properly. To automatically restore the plot
margins, call the function as in the example.
\end{Details}
\begin{Value}
The return value of \samp{par("mar")} when the function was called.
\end{Value}
\begin{Author}\relax
Jim Lemon
\end{Author}
\begin{SeeAlso}\relax
\samp{\LinkA{rect}{rect}}
\end{SeeAlso}
\begin{Examples}
\begin{ExampleCode}
 xy.pop<-c(3.2,3.5,3.6,3.6,3.5,3.5,3.9,3.7,3.9,3.5,3.2,2.8,2.2,1.8,
  1.5,1.3,0.7,0.4)
 xx.pop<-c(3.2,3.4,3.5,3.5,3.5,3.7,4,3.8,3.9,3.6,3.2,2.5,2,1.7,1.5,
  1.3,1,0.8)
 agelabels<-c("0-4","5-9","10-14","15-19","20-24","25-29","30-34",
  "35-39","40-44","45-49","50-54","55-59","60-64","65-69","70-74",
  "75-79","80-44","85+")
 xycol<-color.gradient(c(0,0,0.5,1),c(0,0,0.5,1),c(1,1,0.5,1),18)
 xxcol<-color.gradient(c(1,1,0.5,1),c(0.5,0.5,0.5,1),c(0.5,0.5,0.5,1),18)
 par(mar=pyramid.plot(xy.pop,xx.pop,labels=agelabels,
  main="Australian population pyramid 2002",xycol=xycol,xxcol=xxcol))
 # three column matrices
 avtemp<-c(seq(11,2,by=-1),rep(2:6,each=2),seq(11,2,by=-1))
 malecook<-matrix(avtemp+sample(-2:2,30,TRUE),ncol=3)
 femalecook<-matrix(avtemp+sample(-2:2,30,TRUE),ncol=3)
 # use a background color
 par(bg="#eedd55")
 # group by age
 agegrps<-c("0-10","11-20","21-30","31-40","41-50","51-60",
  "61-70","71-80","81-90","91+")
 oldmar<-pyramid.plot(malecook,femalecook,labels=agegrps,
  unit="Bowls per month",xycol=c("#ff0000","#eeee88","#0000ff"),
  xxcol=c("#ff0000","#eeee88","#0000ff"),
  top.labels=c("Males","Age","Females"),gap=3)
 # put a box around it
 box()
 # give it a title
 mtext("Porridge temperature by age and sex of cook",3,2,cex=1.5)
 # stick in a legend
 legend(par("usr")[1],11,c("Too hot","Just right","Too cold"),
  fill=c("#ff0000","#eeee88","#0000ff"))
 # don't forget to restore the margins and background
 par(mar=oldmar,bg="transparent")
\end{ExampleCode}
\end{Examples}

