\HeaderA{pasteCols}{Paste the columns of a matrix together.}{pasteCols}
\keyword{misc}{pasteCols}
\begin{Description}\relax
Paste the columns of a matrix together to form as
many "words" as there are columns.
\end{Description}
\begin{Usage}
\begin{verbatim}
 pasteCols(x,sep="")
\end{verbatim}
\end{Usage}
\begin{Arguments}
\begin{ldescription}
\item[\code{x}] A matrix.
\item[\code{sep}] The separator to use in the \samp{paste} command.
\end{ldescription}
\end{Arguments}
\begin{Details}\relax
\samp{pasteCols} pastes the columns of a matrix together to form a vector in
which each element is the concatenation of the elements in each of the columns
of the matrix. It is intended for producing identifiers from a matrix returned
by the \samp{combn} function.
\end{Details}
\begin{Value}
A vector of character strings.
\end{Value}
\begin{Author}\relax
Jim Lemon
\end{Author}
\begin{SeeAlso}\relax
\samp{\LinkA{makeIntersectList}{makeIntersectList}}
\end{SeeAlso}
\begin{Examples}
\begin{ExampleCode}
 # create a matrix of the combinations of the first five letters of the
 # alphabet taken two at a time.
 alpha5<-combn(LETTERS[1:5],2,simplify=TRUE)
 pasteCols(alpha5,sep="+")
\end{ExampleCode}
\end{Examples}

