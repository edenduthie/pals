\HeaderA{clplot}{Plot lines with colors determined by values.}{clplot}
\keyword{misc}{clplot}
\begin{Description}\relax
\samp{clplot} displays a plot of lines for which the colors are dependent
upon the x and y values. \samp{clplot} is similar to \samp{color.scale.lines}
except that while the latter calculates a color for each unique value,
\samp{clplot} assigns colors to groups of values within the cutpoints defined
by \samp{levels}.
\end{Description}
\begin{Usage}
\begin{verbatim}
 clplot(x,y,ylab=deparse(substitute(y)),xlab=deparse(substitute(x)),
  levels=seq(min(y)+(max(y)-min(y))/5,max(y)-(max(y)-min(y))/5,length.out=4),
  cols=c("black","blue","green","orange","red"),lty=1,showcuts=FALSE,...)
\end{verbatim}
\end{Usage}
\begin{Arguments}
\begin{ldescription}
\item[\code{x,y}] numeric data vectors.
\item[\code{ylab,xlab}] Labels for the X and Y axes.
\item[\code{levels}] Cut points to assign colors to the values of \samp{x} and
\samp{y}.
\item[\code{cols}] The colors to be assigned.
\item[\code{lty}] The line type.
\item[\code{showcuts}] Whether to show the positions of the cut points.
\item[\code{...}] additional arguments passed to \samp{plot} or \samp{lines}.
\end{ldescription}
\end{Arguments}
\begin{Value}
nil
\end{Value}
\begin{Author}\relax
Carl Witthoft
\end{Author}
\begin{SeeAlso}\relax
\samp{\LinkA{plot}{plot}}
\end{SeeAlso}
\begin{Examples}
\begin{ExampleCode}
 x<-seq(1,100)
 y<-sin(x/5)+x/20
 clplot(x,y,main="Test of clplot")
\end{ExampleCode}
\end{Examples}

