\HeaderA{twoord.plot}{Plot with two ordinates}{twoord.plot}
\keyword{misc}{twoord.plot}
\begin{Description}\relax
Two sets of values are displayed on the same plot with different ordinate
scales on the left and right.
\end{Description}
\begin{Usage}
\begin{verbatim}
 twoord.plot(lx,ly,rx,ry,data=NULL,xlim=NULL,lylim=NULL,rylim=NULL,
 mar=c(5,4,4,4),lcol=1,rcol=2,xlab="",ylab="",rylab="",lpch=1,rpch=2,
 type="b",halfwidth=0.4,...)
\end{verbatim}
\end{Usage}
\begin{Arguments}
\begin{ldescription}
\item[\code{lx,ly,rx,ry}] y and optional x values for the plot
\item[\code{data}] an optional data frame from which to obtain the above values
\item[\code{xlim}] optional x limits as in \samp{plot}
\item[\code{lylim,rylim}] optional y limits for the left and right axes
respectively
\item[\code{mar}] optional margin adjustment, defaults to \samp{c(5,4,4,4)}
\item[\code{lcol,rcol}] colors to distinguish the two sets of values
\item[\code{xlab,ylab}] axis labels as in \samp{plot}
\item[\code{rylab}] label for the right axis
\item[\code{lpch,rpch}] plot symbols to distinguish the two sets of values
\item[\code{type}] as in \samp{plot}
\item[\code{halfwidth}] Half the width of the bars in user units. The bars are
centered on successive integers if no \samp{x} values are supplied.
\item[\code{...}] additional arguments passed to \samp{plot}.
\end{ldescription}
\end{Arguments}
\begin{Details}\relax
\samp{twoord.plot} automates the process of displaying two sets of
values that have different ranges on the same plot. It is principally
useful in illustrating some relationship between the values across the
observations. It is assumed that the \samp{lx} and \samp{rx} values
are at least adjacent, and probably overlapping.

It is best to pass all the arguments \samp{lx, ly, rx, ry}, but the
function will attempt to substitute sensible x values if one or two
are missing.

If at least one of the \samp{type} arguments is "bar", bars will be plotted instead of
points or lines. It is best to plot the bars first (i.e. relative to the left axis)
if the other type is points or lines, as the bars will usually obscure at least some
of the points or lines. Using NA for the color of the bars will partially correct
this. If both types are to be bars, remember to pass somewhat different x values or
the bars will be overlaid.
\end{Details}
\begin{Value}
nil
\end{Value}
\begin{Note}\relax
There are many objections to the use of plots with two different 
ordinate scales, and some of them are even sensible and supported by 
controlled observation. Many of the objections rest on assertions that the 
spatial arrangement of the values plotted will override all other 
evidence. Here are two:

The viewer will assume that the vertical position of the data points 
indicates a quantitative relationship.

To some extent. It is probably not a good idea to have the spatial 
relationship of the points opposed to their numerical relationship. That 
is to say, if one set of values is in the range of 0-10 and the other 
20-100, it is best to arrange the plot so that the latter values are 
not plotted below the former.

The viewer will assume that an intersection of lines indicates an 
intersection of values.

If the visual elements representing values can be arranged to avoid 
intersections, so much the better. Many people have no trouble 
distinguishing which visual elements are linked to which axis as long as 
they are both coded similarly, usually with colors and/or symbols. In the 
special case where there is an underlying relationship between the two 
such as the probability of that value occurring under some conditions, it 
may help to mark the point(s) where this occurs.

It may be useful to consider \samp{gap.plot} as an alternative.
\end{Note}
\begin{Author}\relax
Jim Lemon (thanks to Christophe Dutang for the idea of using bars and lines
in the same plot)
\end{Author}
\begin{SeeAlso}\relax
\samp{\LinkA{plot}{plot}}
\end{SeeAlso}
\begin{Examples}
\begin{ExampleCode}
 twoord.plot(2:10,seq(3,7,by=0.5)+rnorm(9),
  1:15,rev(60:74)+rnorm(15),xlab="Sequence",
  ylab="Ascending values",rylab="Descending values",
  main="Plot with two ordinates - points and lines")
 twoord.plot(2:10,seq(3,7,by=0.5)+rnorm(9),
  1:15,rev(60:74)+rnorm(15),xlab="Sequence",
  ylab="Ascending values",rylab="Descending values",
  main="Plot with two ordinates - bars on the left",
  type=c("bar","l"),lcol=3,rcol=4)
 twoord.plot(2:10,seq(3,7,by=0.5)+rnorm(9),
  1:15,rev(60:74)+rnorm(15),xlab="Sequence",
  ylab="Ascending values",rylab="Descending values",
  main="Plot with two ordinates - bars on the right",
  type=c("b","bar"),lcol=2,rcol=NA,halfwidth=0.2)
\end{ExampleCode}
\end{Examples}

