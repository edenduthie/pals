\HeaderA{dotplot.mtb}{Minitab style dotplots.}{dotplot.mtb}
\keyword{hplot}{dotplot.mtb}
\begin{Description}\relax
Create a dotplot of a data vector in the sense of ``dotplot'' as
used in the Minitab\eqn{\mbox{\copyright}}{} package.
\end{Description}
\begin{Usage}
\begin{verbatim}
dotplot.mtb(x, xlim = NULL, main = NULL, xlab = NULL, ylab = NULL,
            pch = 19, hist = FALSE, yaxis = FALSE, mtbstyle=TRUE)
\end{verbatim}
\end{Usage}
\begin{Arguments}
\begin{ldescription}
\item[\code{x}] A numeric vector. 
\item[\code{xlim}] The x limits of the plot. 
\item[\code{main}] A title for the plot; defaults to blank.
\item[\code{xlab}] A label for the x axis; defaults to blank.
\item[\code{ylab}] A label for the y axis; defaults to blank.
\item[\code{pch}] The plotting symbol for the dots in the plot;
defaults to a solid disc. 
\item[\code{hist}] Logical scalar; should the plot be done ``histogram
style, i.e. using vertical lines rather than stacks
of dots?
\item[\code{yaxis}] Logical scalar; should a y-axis be produced? 
\item[\code{mtbstyle}] Logical scalar; should the dotplot be done in
the ``Minitab'' style?  I.e. should the zero level
be at the vertical midway point? 
\end{ldescription}
\end{Arguments}
\begin{Details}\relax
The result of \code{hist=TRUE} looks less ugly than stacks of
dots for very large data sets.
\end{Details}
\begin{Value}
Nothing.  A plot is produced as a side effect.
\end{Value}
\begin{Section}{Warnings}
This function does something toadally different from the \code{dotplot()}
(now \code{\LinkA{dotchart}{dotchart}()}) function in the graphics package.

The labelling of the \code{y}-axis is device dependent.
\end{Section}
\begin{Author}\relax
Barry Rowlingson
\email{B.Rowlingson@lancaster.ac.uk}
and Rolf Turner
\email{r.turner@auckland.ac.nz}
\end{Author}
\begin{Examples}
\begin{ExampleCode}
## Not run: 
set.seed(42)
x <- rpois(100,10)
dotplot.mtb(x,main="No y-axis.")
dotplot.mtb(x,yaxis=TRUE,main="With y-axis displayed.")
dotplot.mtb(x,hist=TRUE,main="An \"h\" style plot.")
dotplot.mtb(x,xlim=c(4,16),main="With the x-axis limited.")
dotplot.mtb(x,yaxis=TRUE,mtbstyle=FALSE,main="Non-Minitab style.")
dotplot.mtb(x,yaxis=TRUE,xlab="x",ylab="count",
            main="With x and y axis labels.")
## End(Not run)
\end{ExampleCode}
\end{Examples}

