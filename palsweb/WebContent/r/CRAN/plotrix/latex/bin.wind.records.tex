\HeaderA{bin.wind.records}{Classify wind direction and speed records.}{bin.wind.records}
\keyword{misc}{bin.wind.records}
\begin{Description}\relax
Classifies wind direction and speed records into a matrix of
percentages of observations in speed and direction bins.
\end{Description}
\begin{Usage}
\begin{verbatim}
 bin.wind.records(winddir,windspeed,ndir=8,radians=FALSE,
  speed.breaks=c(0,10,20,30))
\end{verbatim}
\end{Usage}
\begin{Arguments}
\begin{ldescription}
\item[\code{winddir}] A vector of wind directions.
\item[\code{windspeed}] A vector of wind speeds corresponding to the above
directions.
\item[\code{ndir}] Number of direction bins in a compass circle.
\item[\code{radians}] Whether wind directions are in radians.
\item[\code{speed.breaks}] Minimum wind speed for each speed bin.
\end{ldescription}
\end{Arguments}
\begin{Details}\relax
\samp{bin.wind.records} bins a number of wind direction and speed
records into a matrix of percentages of observations that can be used to
display a cumulative wind rose with \samp{oz.windrose} The defaults are those
used by the Australian Bureau of Meteorology.
\end{Details}
\begin{Value}
A matrix of percentages in which the rows represent wind speed categories and 
the columns represent wind direction categories.
\end{Value}
\begin{Author}\relax
Jim Lemon
\end{Author}
\begin{SeeAlso}\relax
\samp{\LinkA{oz.windrose}{oz.windrose}}
\end{SeeAlso}
\begin{Examples}
\begin{ExampleCode}
 winddir<-sample(0:360,100,TRUE)
 windspeed<-sample(0:40,100,TRUE)
 bin.wind.records(winddir,windspeed)
\end{ExampleCode}
\end{Examples}

