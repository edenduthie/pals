\HeaderA{axis.break}{Place a "break" mark on an axis}{axis.break}
\keyword{misc}{axis.break}
\begin{Description}\relax
Places a "break" mark on an axis on an existing plot
\end{Description}
\begin{Usage}
\begin{verbatim}
 axis.break(axis=1,breakpos=NULL,bgcol="white",breakcol="black",
  style="slash",brw=0.02)
\end{verbatim}
\end{Usage}
\begin{Arguments}
\begin{ldescription}
\item[\code{axis}] which axis to break
\item[\code{breakpos}] where to place the break in user units
\item[\code{bgcol}] the color of the plot background
\item[\code{breakcol}] the color of the "break" marker
\item[\code{style}] Either \samp{gap}, \samp{slash} or \samp{zigzag}
\item[\code{brw}] break width relative to plot width
\end{ldescription}
\end{Arguments}
\begin{Value}
nil
\end{Value}
\begin{Note}\relax
There is some controversy about the propriety of using discontinuous
coordinates for plotting, and thus axis breaks. Discontinuous coordinates
allow widely separated groups of values or outliers to appear without
devoting too much of the plot to empty space. The major objection seems 
to be that the reader will be misled by assuming continuous coordinates.
The \samp{gap} style that clearly separates the two sections of the plot
is probably best for avoiding this.
\end{Note}
\begin{Author}\relax
Jim Lemon and Ben Bolker
\end{Author}
\begin{SeeAlso}\relax
\samp{\LinkA{gap.plot}{gap.plot}}
\end{SeeAlso}
\begin{Examples}
\begin{ExampleCode}
 plot(3:10,main="Axis break test")
 # put a break at the default axis and position
 axis.break()
 axis.break(2,2.9,style="zigzag")
 twogrp<-c(rnorm(10)+4,rnorm(10)+20)
 gap.plot(twogrp,gap=c(8,16),xlab="Index",ylab="Group values",
  main="Two separated groups with gap axis break",
  col=c(rep(2,10),rep(3,10)),ytics=c(3,5,18,20))
 legend(12,6,c("Low group","High group"),pch=1,col=2:3)
\end{ExampleCode}
\end{Examples}

