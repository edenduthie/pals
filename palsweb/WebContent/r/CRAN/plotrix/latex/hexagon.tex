\HeaderA{hexagon}{Draw a hexagon}{hexagon}
\keyword{misc}{hexagon}
\begin{Description}\relax
Draws a hexagon on the current graphic device.
\end{Description}
\begin{Usage}
\begin{verbatim}
 hexagon(x,y,unitcell=1,col=NA,border="black")
\end{verbatim}
\end{Usage}
\begin{Arguments}
\begin{ldescription}
\item[\code{x,y}] x and y position of the bottom left corner of the square that
would pack into the same space as the hexagon.
\item[\code{unitcell}] The dimension of the side of the abovementioned square.
\item[\code{col}] The color to fill the hexagon - default is no fill.
\item[\code{border}] The color of the perimeter of the hexagon.
\end{ldescription}
\end{Arguments}
\begin{Value}
nil
\end{Value}
\begin{Note}\relax
Draws a hexagon with the same center as a square that would pack into the
same dimensions as the hexagon. That is, given a grid of squares with
alternate rows shifted one half the length of the sides, the hexagons
drawn would be close packed. Its use in the plotrix package is to provide
an alternative unit cell for the \samp{color2D.matplot} function.
\end{Note}
\begin{Author}\relax
Jim Lemon
\end{Author}
\begin{SeeAlso}\relax
\samp{\LinkA{color2D.matplot}{color2D.matplot}}
\end{SeeAlso}

