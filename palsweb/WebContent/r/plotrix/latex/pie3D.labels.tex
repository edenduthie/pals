\HeaderA{pie3D.labels}{Display labels on a 3D pie chart}{pie3D.labels}
\keyword{misc}{pie3D.labels}
\begin{Description}\relax
Displays labels on a 3D pie chart.
\end{Description}
\begin{Usage}
\begin{verbatim}
 pie3D.labels(radialpos,radius=1,height=0.3,theta=pi/6,
  labels,labelcol=par("fg"),labelcex=1.5,minsep=0.3)
\end{verbatim}
\end{Usage}
\begin{Arguments}
\begin{ldescription}
\item[\code{radialpos}] Position of the label in radians
\item[\code{radius}] the radius of the pie in user units
\item[\code{height}] the height of the pie in user units
\item[\code{theta}] The angle of viewing in radians
\item[\code{labels}] The label to display
\item[\code{labelcol}] The color of the labels
\item[\code{labelcex}] The character expansion factor for the labels
\item[\code{minsep}] The minimum angular separation between label positions.
\end{ldescription}
\end{Arguments}
\begin{Details}\relax
\samp{pie3D.label} displays labels on a 3D pie chart. The positions
of the labels are given as angles in radians (usually the bisector of the
pie sectors). As the labels can be passed directly to \samp{\LinkA{pie3D}{pie3D}},
this function would probably not be called by the user.

\samp{pie3D.labels} tries to separate labels that are placed closer than
\samp{minsep} radians. This simple system will handle minor crowding of
labels. If labels are very crowded, capturing the return value of 
\samp{pie3D} and editing the label positions may allow the user to avoid
manually placing labels.
\end{Details}
\begin{Value}
nil
\end{Value}
\begin{Author}\relax
Jim Lemon
\end{Author}
\begin{SeeAlso}\relax
\samp{\LinkA{pie3D}{pie3D}}, \samp{\LinkA{draw.tilted.sector}{draw.tilted.sector}}
\end{SeeAlso}
\begin{Examples}
\begin{ExampleCode}
 pieval<-c(2,4,6,8)
 bisectors<-pie3D(pieval,explode=0.1,main="3D PIE OPINIONS")
 pielabels<-
  c("We hate\n pies","We oppose\n  pies","We don't\n  care","We just love pies")
 pie3D.labels(bisectors,labels=pielabels)
\end{ExampleCode}
\end{Examples}

