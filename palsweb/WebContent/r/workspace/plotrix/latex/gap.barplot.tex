\HeaderA{gap.barplot}{Display a barplot with a gap (missing range) on one axis}{gap.barplot}
\keyword{misc}{gap.barplot}
\begin{Description}\relax
Displays a barplot with a missing range.
\end{Description}
\begin{Usage}
\begin{verbatim}
 gap.barplot(y,gap,xaxlab,xtics,yaxlab,ytics,ylim=NA,xlab=NULL,
  ylab=NULL,horiz=FALSE,col,...)
\end{verbatim}
\end{Usage}
\begin{Arguments}
\begin{ldescription}
\item[\code{y}] data values
\item[\code{gap}] the range of values to be left out
\item[\code{xaxlab}] labels for the x axis ticks
\item[\code{xtics}] position of the x axis ticks
\item[\code{yaxlab}] labels for the y axis ticks
\item[\code{ytics}] position of the y axis ticks
\item[\code{ylim}] optional y limits for the plot
\item[\code{xlab}] label for the x axis
\item[\code{ylab}] label for the y axis
\item[\code{horiz}] whether to have vertical or horizontal bars
\item[\code{col}] color(s) in which to plot the values
\item[\code{...}] arguments passed to \samp{barplot}.
\end{ldescription}
\end{Arguments}
\begin{Details}\relax
Displays a barplot omitting a range of values on the X or Y axis. Typically 
used when there is a relatively large gap in the range of values 
represented as bar heights. See \samp{\LinkA{axis.break}{axis.break}} for a brief 
discussion of plotting on discontinuous coordinates.

If the user does not ask for specific y limits, the function will calculate
limits based on the range of the data values. If passing specific limits, 
remember to subtract the gap from the upper limit.
\end{Details}
\begin{Value}
The center positions of the bars.
\end{Value}
\begin{Author}\relax
Jim Lemon
\end{Author}
\begin{SeeAlso}\relax
\samp{\LinkA{gap.barplot}{gap.barplot}}
\end{SeeAlso}
\begin{Examples}
\begin{ExampleCode}
 twogrp<-c(rnorm(10)+4,rnorm(10)+20)
 gap.barplot(twogrp,gap=c(8,16),xlab="Index",ytics=c(3,6,17,20),
  ylab="Group values",main="Barplot with gap")
 gap.barplot(twogrp,gap=c(8,16),xlab="Index",ytics=c(3,6,17,20),
  ylab="Group values",horiz=TRUE,main="Horizontal barplot with gap")
\end{ExampleCode}
\end{Examples}

