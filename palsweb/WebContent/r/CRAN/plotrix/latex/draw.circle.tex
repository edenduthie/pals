\HeaderA{draw.circle}{Draw a circle.}{draw.circle}
\keyword{misc}{draw.circle}
\begin{Description}\relax
Draws a circle on an existing plot.
\end{Description}
\begin{Usage}
\begin{verbatim}
 draw.circle(x,y,radius,nv=100,border=NULL,col=NA,lty=1,lwd=1)
\end{verbatim}
\end{Usage}
\begin{Arguments}
\begin{ldescription}
\item[\code{x,y}] Coordinates of the center of the circle.
\item[\code{radius}] Radius of the circle in user units.
\item[\code{nv}] Number of vertices to draw the circle.
\item[\code{border}] Color to use for drawing the circumference.
\item[\code{col}] Color to use for filling the circle.
\item[\code{lty}] Line type for the circumference.
\item[\code{lwd}] Line width for the circumference.
\end{ldescription}
\end{Arguments}
\begin{Value}
A list with the x and y coordinates of the points on the circumference.
\end{Value}
\begin{Note}\relax
The principal advantage of \samp{draw.circle} is that it adjusts for 
the aspect ratio of the plot.
\end{Note}
\begin{Author}\relax
Jim Lemon
\end{Author}
\begin{SeeAlso}\relax
\samp{\LinkA{polygon}{polygon}}
\end{SeeAlso}
\begin{Examples}
\begin{ExampleCode}
 plot(1:5,seq(1,10,length=5),type="n",xlab="",ylab="",main="Test draw.circle")
 draw.circle(2,2,0.5,border="purple",lty=1,lwd=1)
 draw.circle(2.5,8,0.6,border="red",lty=3,lwd=3)
 draw.circle(4,3,0.7,border="green",lty=1,lwd=1)
 draw.circle(3.5,7,0.8,border="blue",lty=2,lwd=2)
\end{ExampleCode}
\end{Examples}

