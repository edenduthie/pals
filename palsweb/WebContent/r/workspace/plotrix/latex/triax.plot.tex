\HeaderA{triax.plot}{Triangle plot}{triax.plot}
\keyword{misc}{triax.plot}
\begin{Description}\relax
Display a triangle plot with optional grid.
\end{Description}
\begin{Usage}
\begin{verbatim}
 triax.plot(x=NULL,main="",at=seq(0.1,0.9,by=0.1),
 axis.labels=NULL,tick.labels=NULL,col.axis="black",cex.axis=1,cex.ticks=1,
 align.labels=TRUE,show.grid=FALSE,col.grid="gray",lty.grid=par("lty"),
 cc.axes=FALSE,show.legend=FALSE,label.points=FALSE,point.labels=NULL,
 col.symbols="black",pch=par("pch"),no.add=TRUE,...)
\end{verbatim}
\end{Usage}
\begin{Arguments}
\begin{ldescription}
\item[\code{x}] Matrix where each row is three proportions or percentages
that must sum to 1 or 100 respectively.
\item[\code{main}] The title of the triangle plot. Defaults to nothing.
\item[\code{at}] The tick positions on the three axes.
\item[\code{axis.labels}] Labels for the three axes in the order left, right,
bottom. Defaults to the column names.
\item[\code{tick.labels}] The tick labels for the three axes as a list with
three components l, r and b (left, right and bottom).
Defaults to argument \samp{at} (proportions).
\item[\code{col.axis}] Color of the triangular axes, ticks and labels.
\item[\code{cex.axis}] Character expansion for axis labels.
\item[\code{cex.ticks}] Character expansion for the tick labels.
\item[\code{align.labels}] Logical - whether to align axis and tick labels with
the axes.
\item[\code{show.grid}] Whether to display grid lines at the ticks.
\item[\code{col.grid}] Color of the grid lines. Defaults to gray.
\item[\code{lty.grid}] Type of line for the grid.
\item[\code{cc.axes}] Whether axes and axis ticks should be clockwise or
counterclockwise.
\item[\code{show.legend}] Logical - whether to display a legend.
\item[\code{label.points}] Logical - whether to call \samp{thigmophobe.labels} to
label the points.
\item[\code{point.labels}] Optional labels for the points and/or legend.
\item[\code{col.symbols}] Color of the symbols representing each value.
\item[\code{pch}] Symbols to use in plotting values.
\item[\code{no.add}] Whether to restore the previous plotting parameters
(\samp{TRUE}) or leave them, allowing more points to be added.
\item[\code{...}] Additional arguments passed to \samp{points}.
\end{ldescription}
\end{Arguments}
\begin{Details}\relax
\samp{triax.plot} displays a triangular plot area on which proportions
or percentages are displayed. A grid or legend may also be displayed.
\end{Details}
\begin{Value}
A list containing \samp{xypos} (the \samp{x,y} positions plotted)
and \samp{oldpar} (the plotting parameters at the time \samp{triax.plot}
was called).
\end{Value}
\begin{Note}\relax
A three axis plot can only properly display one or more
sets of three proportions that each sum to 1 (or percentages that sum
to 100). Other values may be scaled to proportions (or percentages), 
but unless each set of three sums to 1 (or 100), they will not plot
properly and \samp{triax.points} will complain appropriately. Note also
that \samp{triax.plot} will only display properly in a square plot,
which is forced by \samp{par(pty="s")}.

In case the user does want to plot values with different sums, the
axis tick labels can be set to different ranges to accomodate this.
\samp{triax.points} will still complain, but it will plot the values.

If planning to add points with \samp{triax.points} call \samp{triax.plot}
with \samp{no.add=FALSE} and restore plotting parameters after the
points are added.
\end{Note}
\begin{Author}\relax
Jim Lemon - thanks to Ben Daughtry for the info on counterclockwise axes.
\end{Author}
\begin{SeeAlso}\relax
\samp{\LinkA{triax.points}{triax.points}},
\samp{\LinkA{triax.abline}{triax.abline}},
\samp{\LinkA{thigmophobe.labels}{thigmophobe.labels}}
\end{SeeAlso}
\begin{Examples}
\begin{ExampleCode}
 data(soils)
 triax.plot(soils[1:10,],main="DEFAULT")
 triax.plot(soils[1:10,],main="PERCENTAGES (Counterclockwise axes)",
  tick.labels=list(l=seq(10,90,by=10),r=seq(10,90,by=10),b=seq(10,90,by=10)),
  pch=3,cc.axes=TRUE)
 triax.return<-triax.plot(soils[1:6,],main="GRID AND LEGEND",
  show.grid=TRUE,show.legend=TRUE,col.symbols=1:6,pch=4)
 # triax.plot changes a few parameters
 par(triax.return$oldpar)
\end{ExampleCode}
\end{Examples}

