\HeaderA{furc}{Plot a dendrite.}{furc}
\keyword{misc}{furc}
\begin{Description}\relax
Plot one level of a dendrogram displaying two or more mutually 
exclusive attributes.
\end{Description}
\begin{Usage}
\begin{verbatim}
 furc(x,xpos,yrange,toplevel,cex=1)
\end{verbatim}
\end{Usage}
\begin{Arguments}
\begin{ldescription}
\item[\code{x}] A \samp{dendrite} object containing the counts of objects having
combinations of mutually exclusive attributes.
\item[\code{xpos}] The horizontal position on the plot to display the current level
of the dendrogram.
\item[\code{yrange}] The range of values in which the current level of the dendrogram
will be displayed.
\item[\code{toplevel}] A flag for the function to know whether it is at the top level
of the dendrogram or not. Do not change this argument.
\item[\code{cex}] The character expansion to use in the display.
\end{ldescription}
\end{Arguments}
\begin{Details}\relax
\samp{furc} displays one \emph{furc}ation of the dendrogram. A furcation is
a single box displaying its label and count that may split into finer
divisions. If so, \samp{furc} calls itself for each furcation until there are
no more splits.
\end{Details}
\begin{Value}
nil
\end{Value}
\begin{Author}\relax
Jim Lemon
\end{Author}
\begin{SeeAlso}\relax
\samp{\LinkA{plot.dendrite}{plot.dendrite}}
\end{SeeAlso}

