\HeaderA{soil.texture.uk}{Soil texture triangle plot using UK conventions}{soil.texture.uk}
\keyword{misc}{soil.texture.uk}
\begin{Description}\relax
Display a UK style soil texture triangle with optional grid, labels and
soil texture points.
\end{Description}
\begin{Usage}
\begin{verbatim}
 soil.texture.uk(soiltexture = NULL, main = "",at = seq(0.1, 0.9, by = 0.1),
  axis.labels = c("percent sand", "percent silt", "percent clay"),
  tick.labels = list(l = seq(10, 90, by = 10), r = seq(10, 90, by = 10),
  b = seq(10, 90, by = 10)), show.names = TRUE,
  show.lines = TRUE, col.names = "gray", bg.names = par("bg"),
  show.grid = FALSE, col.axis = "black", col.lines = "gray",
  col.grid = "gray", lty.grid = 3, show.legend = FALSE, label.points = FALSE,
  point.labels = NULL, col.symbols = "black", pch = par("pch"),
  h1 = NA, h3 = NA, t1 = NA, t3 = NA, lwduk = 2, xpos = NA, ypos = NA,
  snames = NA, cexuk = 1.1, ...)
\end{verbatim}
\end{Usage}
\begin{Arguments}
\begin{ldescription}
\item[\code{soiltexture}] Matrix of soil textures where each row is a
soil sample and three columns containing the percentages of the
components sand, silt and clay in the range 0 to 100.
\item[\code{main}] The title of the soil texture plot. Defaults to nothing.
\item[\code{at}] Positions on the three axes where ticks will be drawn.
\item[\code{axis.labels}] Labels for the axes.
\item[\code{tick.labels}] The tick labels for the three axes.
\item[\code{show.names}] Logical - whether to show the names of different
soil types within the soil triangle.
\item[\code{show.lines}] Logical - whether to show the boundaries of the
different soil types within the soil triangle.
\item[\code{col.names}] Color of the soil names. Defaults to gray.
\item[\code{bg.names}] Color to use when drawing a blank patch for the names
of soil types.
\item[\code{show.grid}] Logical - whether to show grid lines at each 10
level of each soil component.
\item[\code{col.axis}] Color of the triangular axes, ticks and labels.
\item[\code{col.lines}] Color of the boundary lines. Defaults to gray.
\item[\code{col.grid}] Color of the grid lines. Defaults to gray.
\item[\code{lty.grid}] Type of line for the grid. Defaults to dashed.
\item[\code{show.legend}] Logical - whether to display a legend.
\item[\code{label.points}] Logical - whether to call
\samp{\LinkA{thigmophobe.labels}{thigmophobe.labels}} to label the points.
\item[\code{point.labels}] Optional labels for the points or legend.
\item[\code{col.symbols}] Color of the symbols representing each value.
\item[\code{pch}] Symbols to use in plotting values.
\item[\code{h1,h3,t1,t3}] Points used in drawing boundaries for soil types.
\item[\code{lwduk}] Line width for the boundaries
\item[\code{xpos,ypos}] Positions for the soil type labels.
\item[\code{snames}] Soil type labels.
\item[\code{cexuk}] Character expansion for the soil type labels.
\item[\code{...}] Additional arguments passed to \samp{\LinkA{triax.points}{triax.points}}
and then \samp{points}.
\end{ldescription}
\end{Arguments}
\begin{Details}\relax
\samp{soil.texture.uk} displays a triangular plot area on which soil
textures defined as proportions of sand, silt and clay can be plotted.
It is similar to the \samp{soil.texture} function but uses the UK
display conventions.
\end{Details}
\begin{Value}
If \samp{soiltexture} was included, a list of the \samp{x,y}
positions of the soil types plotted. If not, nil.
\end{Value}
\begin{Author}\relax
Julian Stander
\end{Author}
\begin{SeeAlso}\relax
\samp{\LinkA{triax.plot}{triax.plot}}
\end{SeeAlso}
\begin{Examples}
\begin{ExampleCode}
 soils.sw.percent<-data.frame(
  Sand=c(67,67,66,67,36,25,24,59,27,9,8,8,20,
  45,50,56,34,29,39,41,94,98,97,93,96,99),
  Silt=c(17,16,9,8,39,48,54,27,46,70,68,68,66,
  34,30,24,48,53,46,48,2,2,2,4,1,1),
  Clay=c(16,17,25,25,25,27,22,14,27,21,24,24,
  14,21,20,20,18,18,15,11,4,0,1,3,3,0))
 soils.sw.cols <- c(1, 1, 1, 1, 2, 2, 2, 2, 2, 3, 3,
  3, 3, 4, 4, 4, 5, 5, 5, 5, 6, 6, 6, 6, 6, 6)
 soils.sw.names <- c("Ardington","Astrop","Atrim",
  "Banbury","Beacon","Beckfoot")
 soil.texture.uk(soils.sw.percent,
  main = "Ternary Diagram for Some Soils from South West England",
  col.lines = "black", col.names = "black", show.grid = TRUE,
  col.grid = "blue", lty.grid = 2,  pch = 16, cex = 1.0,
  col.symbols = soils.sw.cols, h1 = NA, h3 = NA, t1 = NA,
  t3 = NA , lwduk = 2, xpos = NA, ypos = NA,
  snames = NA, cexuk = 1.1)
 legend("topleft", legend = soils.sw.names, col = 1:max(soils.sw.cols),
  pch = 16, cex = 1.1, title = "Location", bty = "n")
\end{ExampleCode}
\end{Examples}

