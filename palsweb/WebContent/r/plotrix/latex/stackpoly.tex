\HeaderA{stackpoly}{Display the columns of a matrix or data frame as stacked polygons.}{stackpoly}
\keyword{misc}{stackpoly}
\begin{Description}\relax
Plot one or more columns of numeric values as the top edges of polygons
instead of lines.
\end{Description}
\begin{Usage}
\begin{verbatim}
 stackpoly(x,y=NULL,main="",xlab="",ylab="",xat=NA,xaxlab=NA,
  xlim=NA,ylim=NA,lty=1,border=NA,col=NA,staxx=FALSE,stack=FALSE,
  axis4=TRUE,...)
\end{verbatim}
\end{Usage}
\begin{Arguments}
\begin{ldescription}
\item[\code{x}] A numeric data frame or matrix with the \samp{x} values. If
\samp{y} is NULL, these will become the \samp{y} values and the \samp{x}
positions will be the integers from 1 to dim(x)[1].
\item[\code{y}] The \samp{y} values.
\item[\code{main}] The title for the plot.
\item[\code{xlab,ylab}] x and y axis labels for the plot.
\item[\code{xat}] Where to put the optional xaxlabs.
\item[\code{xaxlab}] Optional labels for the x positions.
\item[\code{xlim}] Optional x limits.
\item[\code{ylim}] Optional y limits.
\item[\code{lty}] Line type for the polygon borders.
\item[\code{border}] Color for the polygon borders.
\item[\code{col}] Color to fill the polygons.
\item[\code{staxx}] Whether to call \samp{staxlab} to stagger the x axis labels.
\item[\code{stack}] Whether to stack the successive values on top of each other.
\item[\code{axis4}] Whether to display the right ordinate on the plot.
\item[\code{...}] Additional arguments passed to \samp{plot}.
\end{ldescription}
\end{Arguments}
\begin{Details}\relax
\samp{stackpoly} is similar to a line plot with the area under the
lines filled with color(s). Ideally, each successive set of y values
is greater than the values in the previous set so that the polygons 
form a rising series of crests. If \samp{stack} is TRUE, this is not a
problem unless some values of \samp{x} are negative.

If \samp{x} or \samp{y} is a vector, not a matrix or list, the values will
be displayed as a "waterfall plot".
\end{Details}
\begin{Value}
nil
\end{Value}
\begin{Author}\relax
Jim Lemon and Thomas Petzoldt (waterfall plot option)
\end{Author}
\begin{SeeAlso}\relax
\samp{\LinkA{polygon}{polygon}}
\end{SeeAlso}
\begin{Examples}
\begin{ExampleCode}
 testx<-matrix(abs(rnorm(100)),nrow=10)
 stackpoly(matrix(cumsum(testx),nrow=10),main="Test Stackpoly I",
  xaxlab=c("One","Two","Three","Four","Five",
  "Six","Seven","Eight","Nine","Ten"),border="black",staxx=TRUE)
 stackpoly(testx,main="Test Stackpoly II",
  xaxlab=c("One","Two","Three","Four","Five",
  "Six","Seven","Eight","Nine","Ten"),border="black",
  staxx=TRUE,stack=TRUE)
 stackpoly(rev(sort(testx-mean(testx))),main="Test Waterfall Plot",
  col="green",border="black")
\end{ExampleCode}
\end{Examples}

