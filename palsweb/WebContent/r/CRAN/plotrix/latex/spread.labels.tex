\HeaderA{spread.labels}{Spread labels for irregularly spaced values}{spread.labels}
\keyword{misc}{spread.labels}
\begin{Description}\relax
Places labels for irregularly spaced values in a regular staggered order
\end{Description}
\begin{Usage}
\begin{verbatim}
 spread.labels(x,y,labels=NULL,ony=NA,offsets=NA,between=FALSE,
  linecol=par("fg"),srt=0,...)
\end{verbatim}
\end{Usage}
\begin{Arguments}
\begin{ldescription}
\item[\code{x,y}] x and y data values
\item[\code{labels}] text strings
\item[\code{ony}] Whether to force the labels to be spread horizontally
(FALSE) or vertically (TRUE). Defaults to whichever way the points are
most spread out.
\item[\code{offsets}] How far away from the data points to place the labels.
Defaults to one quarter of the plot span for all, staggered on each side.
\item[\code{between}] Whether to place the labels between two sets of points.
\item[\code{linecol}] Optional colors for the lines drawn to the points.
\item[\code{srt}] Rotation of the labels in degrees.
\item[\code{...}] additional arguments passed to \samp{text}.
\end{ldescription}
\end{Arguments}
\begin{Details}\relax
This function is mainly useful when labeling irregularly spaced data points
that are "spread out" along one dimension. It places the labels regularly
spaced and staggered on the long dimension of the data, drawing lines from
each label to the point it describes.

If \samp{between} is TRUE, the function expects two points for each label
and will attempt to place the labels between two vertical lines of points.
Lines will be drawn from the ends of each label to the two corresponding
points.

If spreading labels horizontally, the user may wish to rotate the labels by
90 degrees (\samp{srt=90}). If long labels run off the edge of the plot,
increase the \samp{xlim} for extra room.
\end{Details}
\begin{Value}
nil
\end{Value}
\begin{Author}\relax
Jim Lemon
\end{Author}
\begin{References}\relax
Cooke, L.J. \& Wardle, J. (2005) Age and gender differences in
children's food preferences. British Journal of Nutrition, 93: 741-746.
\end{References}
\begin{SeeAlso}\relax
\samp{text}
\end{SeeAlso}
\begin{Examples}
\begin{ExampleCode}
 # spread labels out in the x dimension using defaults
 x<-sort(rnorm(10))
 y<-rnorm(10)/10
 plot(x,y,ylim=c(-1,1),type="p")
 nums<-c("one","two","three","four","five","six","seven","eight","nine","ten")
 spread.labels(x,y,nums)
 # food preferences of children by sex (Cooke & Wardle, 2005)
 fpkids<-data.frame(Food=c("Fatty/sugary","Fruit","Starchy","Meat",
  "Proc.meat","Eggs","Fish","Dairy","Vegetables"),
  Female=c(4.21,4.22,3.98,3.57,3.55,3.46,3.34,3.26,3.13),
  Male=c(4.35,4.13,4.02,3.9,3.81,3.64,3.45,3.27,2.96))
 plot(rep(1,9),fpkids$Female,xlim=c(0.8,2.2),
  ylim=range(c(fpkids$Female,fpkids$Male)),xlab="Sex",xaxt="n",
  ylab="Preference rating",main="Children's food preferences by sex",
  col="red")
 axis(1,at=1:2,labels=c("Female","Male"))
 points(rep(2,9),fpkids$Male,col="blue",pch=2)
 spread.labels(rep(1:2,each=9),c(fpkids$Female,fpkids$Male),
  fpkids$Food,between=TRUE,linecol=c("red","blue"))
\end{ExampleCode}
\end{Examples}

