\HeaderA{rescale}{Scale numbers into a new range.}{rescale}
\keyword{misc}{rescale}
\begin{Description}\relax
Scale a vector or matrix of numbers into a new range.
\end{Description}
\begin{Usage}
\begin{verbatim}
 rescale(x,newrange)
\end{verbatim}
\end{Usage}
\begin{Arguments}
\begin{ldescription}
\item[\code{x}] A numeric vector, matrix or data frame.
\item[\code{newrange}] The minimum and maximum value of the range into which
\samp{x} will be scaled.
\end{ldescription}
\end{Arguments}
\begin{Details}\relax
\samp{rescale} performs a simple linear conversion of \samp{x} into the
range specified by \samp{newrange}. Only numeric vectors, matrices or data
frames with some variation will be accepted. NAs are now preserved - 
formerly the function would fail.
\end{Details}
\begin{Value}
On success, the rescaled object, otherwise the original object.
\end{Value}
\begin{Author}\relax
Jim Lemon
\end{Author}
\begin{Examples}
\begin{ExampleCode}
 # scale one vector into the range of another
 normal.counts<-rnorm(100)
 normal.tab<-tabulate(cut(normal.counts,breaks=seq(-3,3,by=1)))
 normal.density<-rescale(dnorm(seq(-3,3,length=100)),range(normal.tab))
 # now plot them
 plot(c(-2.5,-1.5,-0.5,0.5,1.5,2.5),normal.tab,xlab="X values",
  type="h",col="green")
 lines(seq(-3,3,length=100),normal.density,col="blue")
\end{ExampleCode}
\end{Examples}

