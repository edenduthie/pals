\inputencoding{latin1}
\HeaderA{gantt.chart}{Display a Gantt chart}{gantt.chart}
\keyword{misc}{gantt.chart}
\begin{Description}\relax
Displays a Gantt chart with priority coloring
\end{Description}
\begin{Usage}
\begin{verbatim}
 gantt.chart(x=NULL,format="%Y/%m/%d",xlim=NULL,taskcolors=NULL,
  priority.legend=FALSE,vgridpos=NULL,vgridlab=NULL,vgrid.format="%Y/%m/%d",
  half.height=0.25,hgrid=FALSE,main="",xlab="",cylindrical=FALSE)
\end{verbatim}
\end{Usage}
\begin{Arguments}
\begin{ldescription}
\item[\code{x}] a list of task labels, start/end times and task priorities
as returned by \samp{get.gantt.info}. If this is not present,
\samp{\LinkA{get.gantt.info}{get.gantt.info}} will be called.
\item[\code{format}] the format to be used in entering dates/times
(see \samp{\LinkA{strptime}{strptime}}).
\item[\code{xlim}] the horizontal limits of the plot.
\item[\code{taskcolors}] a vector of colors used to illustrate task priority.
\item[\code{priority.legend}] Whether to display a priority color legend.
\item[\code{vgridpos}] optional positions of the vertical grid lines.
\item[\code{vgridlab}] optional labels for the vertical grid lines.
\item[\code{vgrid.format}] format for the vertical grid labels.
\item[\code{half.height}] the proportion of the spacing between task bars that
will be filled by the bar on each side - 0.5 will leave no space.
\item[\code{hgrid}] logical - whether to display grid lines between the bars.
\item[\code{main}] the title of the plot - note that this is actually displayed
using \samp{mtext}.
\item[\code{xlab}] horizontal axis label - usually suppressed.
\item[\code{cylindrical}] Whether to give the bars a cylindrical appearance.
\end{ldescription}
\end{Arguments}
\begin{Details}\relax
If task priority colors are not wanted, set \samp{taskcolors} to a single
value to suppress the coloring. If this is not done, \samp{rainbow} will
be called to generate a different color for each task.

There can now be more than one time interval for each task. If there is,
more than one bar will be displayed for each "task", which may not be a
task at all, but intervals of some attribute for each entity.
\end{Details}
\begin{Value}
The list used to create the chart - see \samp{\LinkA{get.gantt.info}{get.gantt.info}} for
details. This can be saved and reused rather than manually entering the 
information each time the chart is displayed.
\end{Value}
\begin{Author}\relax
Jim Lemon (original by Scott Waichler - features by Ulrike Gromping)
\end{Author}
\begin{SeeAlso}\relax
\samp{\LinkA{get.gantt.info}{get.gantt.info}}
\end{SeeAlso}
\begin{Examples}
\begin{ExampleCode}
 Ymd.format<-"%Y/%m/%d"
 gantt.info<-list(labels=
  c("First task","Second task","Third task","Fourth task","Fifth task"),
  starts=
  as.POSIXct(strptime(
  c("2004/01/01","2004/02/02","2004/03/03","2004/05/05","2004/09/09"),
  format=Ymd.format)),
  ends=
  as.POSIXct(strptime(
  c("2004/03/03","2004/05/05","2004/05/05","2004/08/08","2004/12/12"),
  format=Ymd.format)),
  priorities=c(1,2,3,4,5))
 vgridpos<-as.POSIXct(strptime(c("2004/01/01","2004/02/01","2004/03/01",
  "2004/04/01","2004/05/01","2004/06/01","2004/07/01","2004/08/01",
  "2004/09/01","2004/10/01","2004/11/01","2004/12/01"),format=Ymd.format))
 vgridlab<-
  c("Jan","Feb","Mar","Apr","May","Jun","Jul","Aug","Sep","Oct","Nov","Dec")
 gantt.chart(gantt.info,main="Calendar date Gantt chart (2004)",
  priority.legend=TRUE,vgridpos=vgridpos,vgridlab=vgridlab,hgrid=TRUE)
\end{ExampleCode}
\end{Examples}

