\HeaderA{pie.labels}{Place labels on a pie chart}{pie.labels}
\keyword{misc}{pie.labels}
\begin{Description}\relax
Places labels on a pie chart
\end{Description}
\begin{Usage}
\begin{verbatim}
 pie.labels(x,y,angles,labels,radius=1,bg="white",border=TRUE,...)
\end{verbatim}
\end{Usage}
\begin{Arguments}
\begin{ldescription}
\item[\code{x,y}] x and y position of the center of the pie chart
\item[\code{angles}] A numeric vector representing angles in radians. This is
the return value of \samp{floating.pie}.
\item[\code{labels}] Text strings to label each sector.
\item[\code{radius}] The radius at which to place the labels in user units. The
default is 1.
\item[\code{bg}] The color of the rectangles on which the labels are displayed.
\item[\code{border}] Whether to draw borders around the rectangles.
\item[\code{...}] Arguments passed to \samp{boxed.labels}.
\end{ldescription}
\end{Arguments}
\begin{Value}
nil
\end{Value}
\begin{Note}\relax
Remember that \samp{x} and \samp{y} specify the center of the pie chart and
that the label positions are specified by angles and radii from that 
center.
\end{Note}
\begin{Author}\relax
Jim Lemon
\end{Author}
\begin{SeeAlso}\relax
\samp{\LinkA{floating.pie}{floating.pie}}, \samp{\LinkA{boxed.labels}{boxed.labels}}
\end{SeeAlso}
\begin{Examples}
\begin{ExampleCode}
 pieval<-c(2,4,6,8)
 plot(1:5,type="n",axes=FALSE)
 box()
 bisect.angles<-floating.pie(3,3,pieval)
 pie.labels(3,3,bisect.angles,c("two","four","six","eight"))
\end{ExampleCode}
\end{Examples}

