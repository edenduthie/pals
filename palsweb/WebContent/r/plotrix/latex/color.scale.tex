\HeaderA{color.scale}{Turn values into colors.}{color.scale}
\keyword{misc}{color.scale}
\begin{Description}\relax
Transform numeric values into colors.
\end{Description}
\begin{Usage}
\begin{verbatim}
 color.scale(x,redrange=c(0,1),greenrange=c(0,1),bluerange=c(0,1),
  extremes=NA,na.color=NA)
\end{verbatim}
\end{Usage}
\begin{Arguments}
\begin{ldescription}
\item[\code{x}] a numeric vector, matrix or data frame
\item[\code{redrange,greenrange,bluerange}] color ranges into which 
to scale \samp{x}
\item[\code{extremes}] The colors for the extreme values of \samp{x}.
\item[\code{na.color}] The color to use for NA values of \samp{x}.
\end{ldescription}
\end{Arguments}
\begin{Details}\relax
\samp{color.scale} calculates a sequence of colors by a linear
transformation of the numeric values supplied into the ranges 
for red, green and blue. If only one number is supplied for a
color range, that color remains constant for all values of \samp{x}.
If more than two values are supplied, the \samp{x} values will be
split into equal ranges (one less than the number of colors) and 
the transformation carried out on each range. Values for a color
range must be between 0 and 1.

If \samp{extremes} is not NA, the ranges will be calculated from
its values using \samp{col2rgb}, even if ranges are also supplied.
\samp{extremes} allows the user to just pass the extreme color values
in any format that \samp{col2rgb} will accept.

The user may not want the color scheme to be continuous across some
critical point, often zero. In this case, color scale can be called
separately for the values below and above zero. See the second example
for \samp{color2D.matplot}.
\end{Details}
\begin{Value}
A vector or matrix of hexadecimal color values.
\end{Value}
\begin{Note}\relax
The function is useful for highlighting a numeric dimension or adding
an extra "dimension" to a plot.
\end{Note}
\begin{Author}\relax
Jim Lemon
\end{Author}
\begin{SeeAlso}\relax
\samp{\LinkA{rescale}{rescale}}, \samp{\LinkA{col2rgb}{col2rgb}}
\end{SeeAlso}
\begin{Examples}
\begin{ExampleCode}
 # go from green through yellow to red with no blue
 x<-rnorm(20)
 y<-rnorm(20)
 # use y for the color scale
 plot(x,y,col=color.scale(y,c(0,1,1),c(1,1,0),0),main="Color scale plot",
  pch=16,cex=2)
\end{ExampleCode}
\end{Examples}

