\HeaderA{clock24.plot}{Plot values on a 24 hour "clockface".}{clock24.plot}
\keyword{misc}{clock24.plot}
\begin{Description}\relax
\samp{clock24.plot} displays a plot of radial lines, symbols or a polygon
centered at the midpoint of the plot frame on a 24 hour 'clockface'. 
In contrast to the default behavior of \samp{radial.plot}, the positions 
are interpreted as beginning at vertical (000) and moving clockwise.
\end{Description}
\begin{Usage}
\begin{verbatim}
 clock24.plot(lengths,clock.pos,labels=NULL,label.pos=NULL,rp.type="r",...)
\end{verbatim}
\end{Usage}
\begin{Arguments}
\begin{ldescription}
\item[\code{lengths}] numeric data vector. Magnitudes will be represented as
line lengths, or symbol or polygon vertex positions.
\item[\code{clock.pos}] numeric vector of positions on the 'clockface'.
These must be in decimal hours and will be rescaled to radians.
\item[\code{labels}] Labels to place at the circumference.
\item[\code{label.pos}] Radial positions of the labels.
\item[\code{rp.type}] Whether to plot radial lines, symbols or a polygon.
\item[\code{...}] additional arguments are passed to \samp{radial.plot} and
then to \samp{plot}.
\end{ldescription}
\end{Arguments}
\begin{Value}
nil
\end{Value}
\begin{Author}\relax
Jim Lemon
\end{Author}
\begin{SeeAlso}\relax
\samp{\LinkA{polar.plot}{polar.plot}},\samp{\LinkA{radial.plot}{radial.plot}}
\end{SeeAlso}
\begin{Examples}
\begin{ExampleCode}
 testlen<-rnorm(24)*2+5
 testpos<-0:23+rnorm(24)/4
 clock24.plot(testlen,testpos,main="Test Clock24 (lines)",show.grid=FALSE,
  line.col="green",lwd=3)
 if(dev.interactive()) par(ask=TRUE)
 # now do a 'daylight' plot
 clock24.plot(testlen[7:19],testpos[7:19],
  main="Test Clock24 daytime (symbols)",
  point.col="blue",rp.type="s",lwd=3)
 # reset the margins
 par(mar=c(5,4,4,2))
\end{ExampleCode}
\end{Examples}

