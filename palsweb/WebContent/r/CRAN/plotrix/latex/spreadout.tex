\HeaderA{spreadout}{Spread out a vector of numbers to a minimum interval.}{spreadout}
\keyword{misc}{spreadout}
\begin{Description}\relax
Spread out a vector of numbers so that there is a minimum interval
between any two numbers when in ascending or descending order.
\end{Description}
\begin{Usage}
\begin{verbatim}
 spreadout(x,mindist)
\end{verbatim}
\end{Usage}
\begin{Arguments}
\begin{ldescription}
\item[\code{x}] A numeric vector which may contain NAs.
\item[\code{mindist}] The minimum interval between any two values when in ascending
or descending order.
\end{ldescription}
\end{Arguments}
\begin{Details}\relax
\samp{spreadout} starts at or near the middle of the vector and increases the
intervals between the ordered values. NAs are preserved. \samp{spreadout}
first tries to spread groups of values with intervals less than \samp{mindist}
out neatly away from the mean of the group. If this doesn't entirely succeed,
a second pass that forces values away from the middle is performed.

\samp{spreadout} is currently used to avoid overplotting of axis tick labels
where they may be close together.
\end{Details}
\begin{Value}
On success, the spread out values. If there are less than two valid
values, the original vector is returned.
\end{Value}
\begin{Author}\relax
Jim Lemon
\end{Author}
\begin{Examples}
\begin{ExampleCode}
 spreadout(c(1,3,3,3,3,5),0.2)
 spreadout(c(1,2.5,2.5,3.5,3.5,5),0.2)
 spreadout(c(5,2.5,2.5,NA,3.5,1,3.5,NA),0.2)
 # this will almost always invoke the brute force second pass
 spreadout(rnorm(10),0.5)
\end{ExampleCode}
\end{Examples}

