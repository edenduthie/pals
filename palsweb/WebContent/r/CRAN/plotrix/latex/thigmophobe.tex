\HeaderA{thigmophobe}{Find the direction away from the closest point}{thigmophobe}
\keyword{misc}{thigmophobe}
\begin{Description}\relax
Find the direction away from the closest point
\end{Description}
\begin{Usage}
\begin{verbatim}
 thigmophobe(x,y)
\end{verbatim}
\end{Usage}
\begin{Arguments}
\begin{ldescription}
\item[\code{x,y}] Numeric data vectors. Typically the x/y coordinates of
plotted points. If arrays are passed, they will be silently coerced to
numeric vectors.
\end{ldescription}
\end{Arguments}
\begin{Details}\relax
\samp{thigmophobe} returns the direction (as 1\textbar{}2\textbar{}3\textbar{}4 - see pos= in 
\samp{text}) away from the nearest point to each of the points 
described by \samp{x} and \samp{y}.
\end{Details}
\begin{Value}
A vector of directions away from the point nearest to each point.
\end{Value}
\begin{Note}\relax
Typically used to get the offset to automatically place labels
on a scatterplot or similar using \samp{thigmophobe.labels} to avoid 
overlapping labels. The name means "one who fears being touched".
\end{Note}
\begin{Author}\relax
Jim Lemon - thanks to Gustaf Rydevik for the "names" bug fix
and to Steve Ellison for the suggestion about arrays.
\end{Author}
\begin{SeeAlso}\relax
\samp{\LinkA{thigmophobe.labels}{thigmophobe.labels}}
\end{SeeAlso}
\begin{Examples}
\begin{ExampleCode}
 x<-rnorm(10)
 y<-rnorm(10)
 thigmophobe(x,y)
\end{ExampleCode}
\end{Examples}

