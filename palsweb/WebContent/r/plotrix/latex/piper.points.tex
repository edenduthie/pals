\HeaderA{piper.points}{Display the points on a Piper diagram.}{piper.points}
\keyword{misc}{piper.points}
\begin{Description}\relax
Displays the points for the anions and cations in the Piper
diagram.
\end{Description}
\begin{Usage}
\begin{verbatim}
 piper.points(ca,mg,so4,cl,ions=data.frame(ca=ca,mg=mg,so4=so4,cl=cl),sites,
  ppm=TRUE,chull=FALSE,pch=3,main="",cex.pch=1,col=NA,pch.lwd=0.7,plot.sep=0.15)
\end{verbatim}
\end{Usage}
\begin{Arguments}
\begin{ldescription}
\item[\code{ca,mg,so4,cl}] Concentrations of the ions.
\item[\code{ions}] data frame containing the above values.
\item[\code{sites}] indices for ions.
\item[\code{ppm}] Whether the concentrations are in milligrams/liter (parts per
million) or milliequivalents.
\item[\code{chull}] Whether to display points or polygons for the concentrations.
\item[\code{pch}] Symbol to use in plotting points.
\item[\code{main}] title for the diagram
\item[\code{cex.pch}] Expansion for the points.
\item[\code{col}] Colors for the points/polygons.
\item[\code{pch.lwd}] Line widths.
\item[\code{plot.sep}] The offsets for the ions.
\end{ldescription}
\end{Arguments}
\begin{Details}\relax
\samp{piper.points} displays the points/polygons in the Piper diagram.
\end{Details}
\begin{Value}
A list containing the coordinates of the points and optionally
information about the indices and colors.
\end{Value}
\begin{Author}\relax
Mike Cheetham
\end{Author}
\begin{SeeAlso}\relax
\samp{\LinkA{piper.diagram}{piper.diagram}}
\end{SeeAlso}

