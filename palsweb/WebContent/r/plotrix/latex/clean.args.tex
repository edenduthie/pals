\HeaderA{clean.args}{Remove inappropriate arguments from an argument list}{clean.args}
\aliasA{remove.args}{clean.args}{remove.args}
\keyword{programming}{clean.args}
\begin{Description}\relax
Takes a list of arguments and eliminates those that are not
appropriate for passing to a particular function (and hence
would produce an error if passed).
\end{Description}
\begin{Usage}
\begin{verbatim}
 clean.args(argstr,fn,exclude.repeats=FALSE,exclude.other=NULL,dots.ok=TRUE)
 remove.args(argstr,fn)
\end{verbatim}
\end{Usage}
\begin{Arguments}
\begin{ldescription}
\item[\code{argstr}] a named list of arguments, e.g. from \samp{\dots}
\item[\code{fn}] a function
\item[\code{exclude.repeats}] (logical) remove repeated arguments?
\item[\code{exclude.other}] a character vector of names of additional arguments to remove
\item[\code{dots.ok}] should "..." be allowed in the argument list?
\end{ldescription}
\end{Arguments}
\begin{Value}
\samp{clean.args} returns a list which is a copy of \samp{argstr} with
arguments inappropriate for \samp{fn} removed; \samp{remove.args}
removes the arguments for \samp{fn} from the list.
\end{Value}
\begin{Author}\relax
Ben Bolker
\end{Author}
\begin{Examples}
\begin{ExampleCode}
 tststr <- list(n=2,mean=0,sd=1,foo=4,bar=6) 
 clean.args(tststr,rnorm)
 try(do.call("rnorm",tststr))
 do.call("rnorm",clean.args(tststr,rnorm))
 remove.args(tststr,rnorm)
 ## add example of combining arg. lists?
\end{ExampleCode}
\end{Examples}

