\HeaderA{pie3D}{Display a 3D pie chart}{pie3D}
\keyword{misc}{pie3D}
\begin{Description}\relax
Displays a 3D pie chart with optional labels.
\end{Description}
\begin{Usage}
\begin{verbatim}
 pie3D(x,edges=100,radius=1,height=0.3,theta=pi/6,start=0,border=par("fg"),
  col=NULL,labels=NULL,labelpos=NULL,labelcol=par("fg"),labelcex=1.5,
  sector.order=NULL,explode=0,shade=0.8,...)
\end{verbatim}
\end{Usage}
\begin{Arguments}
\begin{ldescription}
\item[\code{x}] a numeric vector for which each value will be a sector
\item[\code{edges}] the number of lines forming an ellipse
\item[\code{radius}] the radius of the pie in user units
\item[\code{height}] the height of the pie in user units
\item[\code{theta}] The angle of viewing in radians
\item[\code{start}] The angle at which to start drawing sectors.
\item[\code{border}] The color of the sector border lines
\item[\code{col}] The colors of the sectors
\item[\code{labels}] Optional labels for each sector
\item[\code{labelpos}] Optional positions for the labels
\item[\code{labelcol}] The color of the labels
\item[\code{labelcex}] The character expansion factor for the labels
\item[\code{sector.order}] Allows the operator to specify the order in
which the sectors are drawn.
\item[\code{explode}] The amount to "explode" the pie in user units
\item[\code{shade}] If \textgreater{} 0 and \textless{} 1, the proportion to reduce the
brightness of the sector color to get a better 3D effect.
\item[\code{...}] graphical parameters passed to \samp{plot}
\end{ldescription}
\end{Arguments}
\begin{Details}\relax
\samp{pie3D} scales the values in \samp{x} so that they total 2*pi,
dropping zeros and NAs. It then displays an empty plot, calculates
the sequence for drawing the sectors and calls \samp{draw.tilted.sector}
to draw each sector. If labels are supplied, it will call \samp{pie3D.label}
to place a label for each sector. If supplied, the number of labels, label 
positions and sector colors must be at least equal to the number of values 
in \samp{x}. If the labels are long, it may help to reduce the radius of
the pie as in the example below.
\end{Details}
\begin{Value}
The bisecting angle of the sectors in radians.
\end{Value}
\begin{Note}\relax
Due to the somewhat primitive method used to draw sectors, a sector that
extends beyond both pi/2 and 3*pi/2 radians in either direction may not
display properly. Setting \samp{start} to pi/2 will often fix this, but
the user may have to adjust \samp{start} and the order of sectors in extreme
cases. The argument \samp{sector.order} allows the user to specify a vector
of integers that will override the calculation of the order in which the
sectors are drawn. This is usually necessary when a very large sector that 
extends past 3*pi/2 is overlapped by a smaller sector next to it.

While \samp{pie3D} can be used to display a 2D pie chart by setting
height=0 and theta=pi, the labels produced by \samp{pie3D.labels} will
not be well positioned. It is probably better to use \samp{floating.pie}
for this, or use \samp{pie.labels} for the labels.
\end{Note}
\begin{Author}\relax
Jim Lemon
\end{Author}
\begin{SeeAlso}\relax
\samp{\LinkA{pie3D.labels}{pie3D.labels}}, \samp{\LinkA{draw.tilted.sector}{draw.tilted.sector}}
\end{SeeAlso}
\begin{Examples}
\begin{ExampleCode}
 pieval<-c(2,4,6,8)
 pielabels<-
  c("We hate\n pies","We oppose\n  pies","We don't\n  care","We just love pies")
 pie3D(pieval,radius=0.9,labels=pielabels,explode=0.1,main="3D PIE OPINIONS")
\end{ExampleCode}
\end{Examples}

