\HeaderA{gap.boxplot}{Display a boxplot with a gap (missing range)}{gap.boxplot}
\keyword{misc}{gap.boxplot}
\begin{Description}\relax
Displays a boxplot with a missing range.
\end{Description}
\begin{Usage}
\begin{verbatim}
 gap.boxplot(x,...,gap=list(top=c(NA,NA),bottom=c(NA,NA)),
 range=1.5,width=NULL,varwidth=FALSE,notch=FALSE,outline=TRUE,
 names,ylim=NA,plot=TRUE,border=par("fg"),col=NULL,log="",
 axis.labels=NULL,pars=list(boxwex=0.8,staplewex=0.5,outwex=0.5),
 horizontal=FALSE,add=FALSE,at=NULL,main=NULL)
\end{verbatim}
\end{Usage}
\begin{Arguments}
\begin{ldescription}
\item[\code{x}] numeric vector or a list of vectors
\item[\code{...}] arguments passed to \samp{\LinkA{boxplot}{boxplot}}.
\item[\code{gap}] the range(s) to be omitted - a list with two components,
\samp{top} and \samp{bottom} each specifying a range to omit. The
default range of \samp{c(NA,NA)} means no omitted range
\item[\code{range}] how far to extend the whiskers, (see \samp{\LinkA{boxplot}{boxplot}})
\item[\code{width}] the relative widths of the boxes
\item[\code{varwidth}] if TRUE, box widths are proportional to the square roots
of the number of observations
\item[\code{notch}] whether to display the confidence intervals for the
median as notches
\item[\code{outline}] whether to display outliers
\item[\code{names}] optional names to display beneath each boxplot
\item[\code{ylim}] Optional y axis limits for the plot.
\item[\code{boxwex}] scale factor for box widths
\item[\code{staplewex}] staple width proportional to box width
\item[\code{outwex}] outlier line width
\item[\code{plot}] dummy argument for consistency with \samp{boxplot} - always
plots
\item[\code{border}] optional color(s) for the box lines
\item[\code{col}] optional color(s) to fill the boxes
\item[\code{log}] whether to use a log scale - currently does nothing
\item[\code{axis.labels}] Optional axis labels.
\item[\code{pars}] optional parameters for consistency with \samp{boxplot}
\item[\code{horizontal}] whether to plot horizontal boxplots - currently
does nothing
\item[\code{add}] whether to add the boxplot(s) to a current plot - currently
does nothing
\item[\code{at}] optional horizontal locations for the boxplots - currently
does nothing
\item[\code{main}] a title for the plot
\end{ldescription}
\end{Arguments}
\begin{Details}\relax
Displays boxplot(s) omitting range(s) of values on the top and/or bottom
of the plot. Typically used when there are outliers far from the boxes.
See \samp{\LinkA{boxplot}{boxplot}} for more detailed descriptions of the
arguments. If the gaps specified include any of the values in the \samp{stats}
matrix returned from \samp{boxplot}, the function will exit with an
error message. This prevents generation of NAs in indexing operations,
which would fail anyway. A gap can include part of a box, but it is unlikely 
that this would be intended by the user.

See \samp{\LinkA{axis.break}{axis.break}} for a brief discussion 
of plotting on discontinuous coordinates.
\end{Details}
\begin{Value}
A list with the same structure as returned by \samp{boxplot}, except that
the values of elements beyond the gap(s) have their true positions on the
plot rather than the original values.
\end{Value}
\begin{Author}\relax
Jim Lemon
\end{Author}
\begin{SeeAlso}\relax
\samp{\LinkA{gap.barplot}{gap.barplot}},\samp{\LinkA{gap.plot}{gap.plot}}
\end{SeeAlso}
\begin{Examples}
\begin{ExampleCode}
 twovec<-list(vec1=c(rnorm(30),-6),vec2=c(sample(1:10,40,TRUE),20))
 gap.boxplot(twovec,gap=list(top=c(12,18),bottom=c(-5,-3)),
 main="Show outliers separately")
 if(dev.interactive()) par(ask=TRUE)
 gap.boxplot(twovec,gap=list(top=c(12,18),bottom=c(-5,-3)),range=0,
 main="Include outliers in whiskers")
 par(ask=FALSE)
\end{ExampleCode}
\end{Examples}

