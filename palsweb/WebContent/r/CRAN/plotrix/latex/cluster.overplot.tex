\HeaderA{cluster.overplot}{Shift overlying points into clusters.}{cluster.overplot}
\keyword{misc}{cluster.overplot}
\begin{Description}\relax
\samp{cluster.overplot} checks for overlying points in the x and y
coordinates passed. Those points that are overlying are moved to form
a small cluster of up to nine points. For large numbers of overlying
points, see \samp{\LinkA{count.overplot}{count.overplot}} or \samp{\LinkA{sizeplot}{sizeplot}}.
If you are unsure of the number of overplots in your data, run
\samp{count.overplot} first to see if there are any potential clusters
larger than nine.
\end{Description}
\begin{Usage}
\begin{verbatim}
 cluster.overplot(x,y,away=NULL,tol=NULL,...)
\end{verbatim}
\end{Usage}
\begin{Arguments}
\begin{ldescription}
\item[\code{x,y}] Numeric data vectors or the first two columns of a matrix
or data frame. Typically the x/y coordinates of points to be plotted.
\item[\code{away}] How far to move overlying points in user units. Defaults to
the width of a lower case "o" in the x direction and 5/8 of the
height of a lower case "o" in the y direction.
\item[\code{tol}] The largest distance between points that will be considered
to be overlying. Defaults to 1/2 of the width of a lower case "o" in 
the x direction and 1/2 of the height of a lower case "o" in the y 
direction.
\item[\code{...}] additional arguments returned as they are passed.
\end{ldescription}
\end{Arguments}
\begin{Value}
A list with two components. For unique x-y pairs the elements will be 
the same as in the original. For overlying points up to eight additional 
points will be generated that will create a cluster of points instead of one.
\end{Value}
\begin{Author}\relax
Jim Lemon
\end{Author}
\begin{SeeAlso}\relax
\samp{\LinkA{count.overplot}{count.overplot}},\samp{\LinkA{sizeplot}{sizeplot}}
\end{SeeAlso}
\begin{Examples}
\begin{ExampleCode}
 xy.mat<-cbind(sample(1:10,200,TRUE),sample(1:10,200,TRUE))
 clusteredpoints<-
  cluster.overplot(xy.mat,col=rep(c("red","green"),each=100))
 plot(clusteredpoints,col=clusteredpoints$col,
  main="Cluster overplot test")
\end{ExampleCode}
\end{Examples}

