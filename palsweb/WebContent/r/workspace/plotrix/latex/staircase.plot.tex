\HeaderA{staircase.plot}{Display a staircase plot.}{staircase.plot}
\keyword{misc}{staircase.plot}
\begin{Description}\relax
Displays a plot showing a sequence of changing totals and increments as
successive linked bars.
\end{Description}
\begin{Usage}
\begin{verbatim}
 staircase.plot(heights,totals=NA,labels=NULL,halfwidth=0.3,main="",
 mar=NA,total.col="blue",inc.col=NA,bg.col=NA,direction="e",las=1,
 display.height=TRUE,stagger=FALSE,...)
\end{verbatim}
\end{Usage}
\begin{Arguments}
\begin{ldescription}
\item[\code{heights}] vector of numeric values or a matrix or data frame with at least
two columns. The first column must be numeric and the second may be numeric or
logical.
\item[\code{totals}] A vector of logicals or zero/non-zero values indicating whether
the corresponding height is a total (TRUE) or an increment (FALSE).
\item[\code{labels}] An optional vector of labels for the bars.
\item[\code{halfwidth}] Half of the width of a bar as a proportion. See Details.
\item[\code{main}] A title for the plot.
\item[\code{mar}] Margins for the plot. Defaults to 10 on the baseline axis, 3 on the
top and 1 on the other two sides.
\item[\code{total.col}] Color(s) for the bars representing successive totals.
\item[\code{inc.col}] Color(s) for the bars representing increments.
\item[\code{bg.col}] The background color for the plot.
\item[\code{direction}] Direction in which the bars should be presented. See Details.
\item[\code{las}] Orientation for the bar labels. See \samp{par}.
\item[\code{display.height}] Whether to display the totals and increments at the upper
ends of the bars. Defaults to TRUE.
\item[\code{stagger}] Whether to stagger the labels to avoid overlap.
\item[\code{...}] arguments passed to \samp{plot}.
\end{ldescription}
\end{Arguments}
\begin{Details}\relax
Displays a plot representing successive changes in counts or values. For example,
if a research study attempts to contact a certain number of people and some
cannot be contacted, some decline to participate, some are ineligible, the final
sample will be smaller than the initial contact list. The first value will be the
total of attempts, there will be a number of decrements, and the last value will
be the actual sample. There may be intermediate totals specified. This produces
a visual display of the sampling procedure. See the example.

The bars are placed at integer values on the axis representing the succession of
counts or values. The width of the bars is determined by the argument
\samp{halfwidth}. This defaults to 0.3, meaning that the bar extends 0.3 to each
side, so that the proportion of bar to space is 0.6 to 0.4. The succession of
bars is determined by the \samp{direction} argument. The default is "e" (east),
meaning that the first bar is at the left of the plot and subsequent bars are
placed to the right. The other three possibilities follow the conventional
compass layout.

The \samp{getFigCtr} function is called to center the plot title in the figure
region as the plot area is typically off center.
\end{Details}
\begin{Value}
nil
\end{Value}
\begin{Author}\relax
Jim Lemon
\end{Author}
\begin{SeeAlso}\relax
\samp{\LinkA{plot}{plot}}, \samp{\LinkA{getFigCtr}{getFigCtr}}
\end{SeeAlso}
\begin{Examples}
\begin{ExampleCode}
 sample_size<-c(500,-72,428,-94,334,-45,289)
 totals<-c(TRUE,FALSE,TRUE,FALSE,TRUE,FALSE,TRUE)
 labels<-c("Contact list","Uncontactable","","Declined","","Ineligible",
  "Final sample")
 staircase.plot(sample_size,totals,labels,main="Acquisition of the sample",
  total.col="gray",inc.col=2:4,bg.col="#eeeebb",direction="s")
\end{ExampleCode}
\end{Examples}

