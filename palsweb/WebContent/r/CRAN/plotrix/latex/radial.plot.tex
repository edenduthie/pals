\HeaderA{radial.plot}{Plot values on a circular grid of 0 to 2*pi radians.}{radial.plot}
\keyword{misc}{radial.plot}
\begin{Description}\relax
Plot numeric values as distances from the center of a circular field in the
directions defined by angles in radians.
\end{Description}
\begin{Usage}
\begin{verbatim}
 radial.plot(lengths,radial.pos=NULL,labels,label.pos=NULL,radlab=FALSE,
 start=0,clockwise=FALSE,rp.type="r",label.prop=1.1,main="",xlab="",ylab="",
 line.col=par("fg"),lty=par("lty"),lwd=par("lwd"),mar=c(2,2,3,2),
 show.grid=TRUE,show.grid.labels=TRUE,show.radial.grid=TRUE,
 grid.col="gray",grid.bg="transparent",
 grid.left=FALSE,grid.unit=NULL,point.symbols=NULL,point.col=NULL,
 show.centroid=FALSE,radial.lim=NULL,radial.labels=NULL,poly.col=NULL,...)
\end{verbatim}
\end{Usage}
\begin{Arguments}
\begin{ldescription}
\item[\code{lengths}] A numeric data vector or matrix. If \samp{lengths}
is a matrix, the rows will be considered separate data vectors.
\item[\code{radial.pos}] A numeric vector or matrix of positions in radians. 
These are interpreted as beginning at the right (0 radians) and moving
counterclockwise. If \samp{radial.pos} is a matrix, the rows must 
correspond to rows of \samp{lengths}.
\item[\code{labels}] Character strings to be placed at the outer ends of
the lines. If set to NA, will suppress printing of labels,
but if missing, the radial positions will be used.
\item[\code{label.pos}] The positions of the labels around the plot in radians.
\item[\code{radlab}] Whether to rotate the outer labels to a radial orientation.
\item[\code{start}] Where to place the starting (zero) point. Defaults to the
3 o'clock position.
\item[\code{clockwise}] Whether to interpret positive positions as clockwise from
the starting point. The default is counterclockwise.
\item[\code{rp.type}] Whether to draw (r)adial lines, a (p)olygon, (s)ymbols
or some combination of these. If \samp{lengths} is a matrix and rp.type is
a vector, each row of \samp{lengths} can be displayed differently.
\item[\code{label.prop}] The label position radius as a proportion of the 
maximum line length.
\item[\code{main}] The title for the plot.
\item[\code{xlab,ylab}] Normally x and y axis labels are suppressed.
\item[\code{line.col}] The color of the radial lines or polygons drawn.
\item[\code{lty}] The line type(s) to be used for polygons or radial lines.
\item[\code{lwd}] The line width(s) to be used for polygons or radial lines.
\item[\code{mar}] Margins for the plot. Allows the user to leave space for
legends, long labels, etc.
\item[\code{show.grid}] Logical - whether to draw a circular grid.
\item[\code{show.grid.labels}] Logical - whether to display labels for the grid.
\item[\code{show.radial.grid}] Whether to draw radial lines to the plot labels.
\item[\code{grid.col}] Color of the circular grid.
\item[\code{grid.bg}] Fill color of above.
\item[\code{grid.left}] Whether to place the radial grid labels on the left side.
\item[\code{grid.unit}] Optional unit description for the grid.
\item[\code{point.symbols}] The symbols for plotting (as in pch).
\item[\code{point.col}] Colors for the symbols.
\item[\code{show.centroid}] Whether to display a centroid.
\item[\code{radial.lim}] The range of the grid circle. Defaults to
\samp{pretty(range(lengths))}, but if more than two values are passed, the
exact values will be displayed.
\item[\code{radial.labels}] Optional labels for the radial grid. The default is
the values of radial.lim.
\item[\code{poly.col}] Fill color if polygons are drawn. Use NA for no fill.
\item[\code{...}] Additional arguments are passed to \samp{plot}.
\end{ldescription}
\end{Arguments}
\begin{Details}\relax
\samp{radial.plot} displays a plot of radial lines, polygon(s),
symbols or a combination of these centered at the midpoint of
the plot frame, the lengths, vertices or positions corresponding
to the numeric magnitudes of the data values. If \samp{show.centroid}
is TRUE, an enlarged point at the centroid of values is displayed. The
centroid is calculated as the average of x and y values unless 
\samp{rp.type="p"}. In this case, the barycenter of the polygon is
calculated. Make sure that these suit your purpose, otherwise calculate
the centroid that you really want and add it with the \samp{points} function.

If the user wants to plot several sets of lines, points or symbols by
passing matrices or data frames of \samp{lengths} and \samp{radial.pos},
remember that these will be grouped by row, so transpose if the data are
grouped by columns.

The size of the labels on the outside of the plot can be adjusted by 
setting \samp{par(cex.axis=)} and that of the labels inside by setting
\samp{par(cex.lab=)}. If \samp{radlab} is TRUE, the labels will be rotated
to a radial alignment. This may help when there are many values and labels.
If some labels are still crowded, try running \samp{label.pos} through the
\samp{spreadout} function.

The radial.plot family of plots is useful for illustrating
cyclic data such as wind direction or speed (but see \samp{oz.windrose}
for both), activity at different times of the day, and so on. While 
\samp{radial.plot} actually does the plotting, another function is usually 
called for specific types of cyclic data. Note that if the observations
are not taken at equal intervals around the circle, the centroid may not
mean much.
\end{Details}
\begin{Value}
nil
\end{Value}
\begin{Author}\relax
Jim Lemon - thanks to Jeremy Claisse and Antonio Hernandez Matias
for the \samp{lty} and \samp{rp.type} suggestions respectively, Patrick
Baker for the request that led to \samp{radlab} and Thomas Steiner for
the request for the \samp{radial.lim} and \samp{radial.labels} modifications.
\end{Author}
\begin{SeeAlso}\relax
\samp{\LinkA{polar.plot}{polar.plot}},\samp{\LinkA{clock24.plot}{clock24.plot}}
\end{SeeAlso}
\begin{Examples}
\begin{ExampleCode}
 testlen<-rnorm(10)*2+5
 testpos<-seq(0,18*pi/10,length=10)
 testlab<-letters[1:10]
 oldpar<-radial.plot(testlen,testpos,main="Test Radial Lines",line.col="red",lwd=3)
 testlen<-c(sin(seq(0,1.98*pi,length=100))+2+rnorm(100)/10)
 testpos<-seq(0,1.98*pi,length=100)
 radial.plot(testlen,testpos,rp.type="p",main="Test Polygon",line.col="blue")
 # now do a 12 o'clock start with clockwise positive
 radial.plot(testlen,testpos,start=pi/2,clockwise=TRUE,
  rp.type="s",main="Test Symbols (clockwise)",
  point.symbols=16,point.col="green",show.centroid=TRUE)
 # one without the circular grid and multiple polygons
 # see the "diamondplot" function for variation on this
 posmat<-matrix(sample(2:9,30,TRUE),nrow=3)
 radial.plot(posmat,labels=paste("X",1:10,sep=""),rp.type="p",
  main="Spiderweb plot",line.col=2:4,show.grid=FALSE,lwd=1:3,
  radial.lim=c(0,10))
 # dissolved ions in water
 ions<-c(3.2,5,1,3.1,2.1,4.5)
 ion.names<-c("Na","Ca","Mg","Cl","HCO3","SO4")
 radial.plot(ions,labels=ion.names,rp.type="p",main="Dissolved ions in water",
  grid.unit="meq/l",radial.lim=c(0,5),poly.col="yellow")
 par(xpd=oldpar$xpd,mar=oldpar$mar,pty=oldpar$pty)
 # reset the margins
 par(mar=c(5,4,4,2))
\end{ExampleCode}
\end{Examples}

