\HeaderA{p.ions}{Calculate ion/total(ion) ratios.}{p.ions}
\keyword{misc}{p.ions}
\begin{Description}\relax
Calculates the ion/total(ions) ratios of ions used in the Piper
diagram.
\end{Description}
\begin{Usage}
\begin{verbatim}
 p.ions(meq,short.output=TRUE)
\end{verbatim}
\end{Usage}
\begin{Arguments}
\begin{ldescription}
\item[\code{meq}] A data frame containing the concentrations of ions used in the
Piper diagram in milliequivalents/liter.
\item[\code{short.output}] Whether to return the proportions of four ions (TRUE)
or include carbonate-bicarbonate and sodium-potassium (FALSE).
\end{ldescription}
\end{Arguments}
\begin{Details}\relax
\samp{p.ions} calculates the proportions of anions or cations to total anions
or cations for the ions used in the Piper diagram.
\end{Details}
\begin{Value}
The proportion of the ions.
\end{Value}
\begin{Author}\relax
Mike Cheetham
\end{Author}
\begin{SeeAlso}\relax
\samp{\LinkA{piper.diagram}{piper.diagram}}
\end{SeeAlso}

