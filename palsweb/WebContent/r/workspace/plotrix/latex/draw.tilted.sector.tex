\HeaderA{draw.tilted.sector}{Display a 3D pie sector}{draw.tilted.sector}
\keyword{misc}{draw.tilted.sector}
\begin{Description}\relax
Displays a 3D pie sector.
\end{Description}
\begin{Usage}
\begin{verbatim}
 draw.tilted.sector(x=0,y=0,edges=100,radius=1,height=0.3,theta=pi/6,
  start=0,end=pi*2,border=par("fg"),col=par("bg"),explode=0,shade=0.8)
\end{verbatim}
\end{Usage}
\begin{Arguments}
\begin{ldescription}
\item[\code{x,y}] Position of the center of the pie sector in user units
\item[\code{edges}] Number of edges to draw a complete ellipse
\item[\code{radius}] the radius of the pie in user units
\item[\code{height}] the height of the pie in user units
\item[\code{theta}] The angle of viewing in radians
\item[\code{start}] Starting angle of the sector
\item[\code{end}] Ending angle of the sector
\item[\code{border}] The color of the sector border lines
\item[\code{col}] Color of the sector
\item[\code{explode}] How far to "explode" the sectors in user units
\item[\code{shade}] If \textgreater{} 0 and \textless{} 1, the proportion to reduce the
brightness of the sector color to get a better 3D effect.
\end{ldescription}
\end{Arguments}
\begin{Details}\relax
\samp{draw.tilted.sector} displays a single 3D pie sector. It is probably
only useful when called from \samp{\LinkA{pie3D}{pie3D}}. The \samp{shade}
argument proportionately reduces the brightness of the RGB color of
the sector to produce a top lighted effect.
\end{Details}
\begin{Value}
The bisector of the pie sector in radians.
\end{Value}
\begin{Author}\relax
Jim Lemon
\end{Author}
\begin{SeeAlso}\relax
\samp{\LinkA{pie3D}{pie3D}}
\end{SeeAlso}

