\HeaderA{gap.plot}{Display a plot with one or two gaps (missing ranges) on one axis}{gap.plot}
\keyword{misc}{gap.plot}
\begin{Description}\relax
Displays a plot with one or two missing ranges on one of the axes.
\end{Description}
\begin{Usage}
\begin{verbatim}
 gap.plot(x,y,gap,gap.axis="y",bgcol="white",breakcol="black",brw=0.02,
  xlim,ylim,xticlab,xtics=NA,yticlab,ytics=NA,lty=rep(1,length(x)),
  col=rep(par("col"),length(x)),pch=rep(1,length(x)),add=FALSE,...)
\end{verbatim}
\end{Usage}
\begin{Arguments}
\begin{ldescription}
\item[\code{x,y}] data values
\item[\code{gap}] the range(s) of values to be left out
\item[\code{gap.axis}] whether the gaps are to be on the x or y axis
\item[\code{bgcol}] the color of the plot background
\item[\code{breakcol}] the color of the "break" marker
\item[\code{brw}] break width relative to plot width
\item[\code{xlim,ylim}] the plot limits.
\item[\code{xticlab}] labels for the x axis ticks
\item[\code{xtics}] position of the x axis ticks
\item[\code{yticlab}] labels for the y axis ticks
\item[\code{ytics}] position of the y axis ticks
\item[\code{lty}] line type(s) to use if there are lines
\item[\code{col}] color(s) in which to plot the values
\item[\code{pch}] symbols to use in plotting.
\item[\code{add}] whether to add values to an existing plot.
\item[\code{...}] arguments passed to \samp{plot} and \samp{points}.
\end{ldescription}
\end{Arguments}
\begin{Details}\relax
Displays a plot omitting one or two ranges of values on one axis. 
Typically used when there is a relatively large gap or two in the overall 
range of one set of values, often because of outliers. The function warns
the user if any values have been omitted by being in the "gap". See 
\samp{\LinkA{axis.break}{axis.break}} for a brief discussion of plotting on 
discontinuous coordinates.

To add more data series to a gap plot, call \samp{gap.plot} with
\samp{add = TRUE}. The same \samp{gap} and \samp{gap.axis} arguments as
in the initial call must be passed or the data will not be displayed
correctly. Remember to pass an explicit \samp{xlim} or \samp{ylim} to
the initial call if the added data exceed the range of the data
initially displayed. Also remember to subtract the width(s) of the gap(s)
if you are passing an explicit \samp{xlim} or \samp{ylim}.

Because the gaps take up some space, it is possible to have a data value
that is just below a gap plotted in the gap. The answer is to make the lower
gap limit a little higher if this is a problem.

If at least four values are passed in \samp{gap}, the first four will be
used to calculate two "gaps" in the plot instead of one. The function does
not check whether these values are sensible, so it is quite easy to
ask for a very silly plot.

The default ticks are usually not ideal, and most users will want to pass
their own tick positions and perhaps labels. Note that \samp{lines} appears
to use only the first \samp{col} and \samp{lty} argument value, so if you
must have lines with different colors and types, use \samp{add=TRUE} and add
them separately (see the third example for the problem and the solution).
\end{Details}
\begin{Value}
nil
\end{Value}
\begin{Author}\relax
Jim Lemon and Ben Bolker (thanks to Zheng Lu for the "add" idea,
and Art Roberts for helping to get the gaps right.)
\end{Author}
\begin{SeeAlso}\relax
\samp{\LinkA{gap.barplot}{gap.barplot}}, \samp{\LinkA{axis.break}{axis.break}}
\end{SeeAlso}
\begin{Examples}
\begin{ExampleCode}
 twogrp<-c(rnorm(5)+4,rnorm(5)+20,rnorm(5)+5,rnorm(5)+22)
 gpcol<-c(2,2,2,2,2,3,3,3,3,3,4,4,4,4,4,5,5,5,5,5)
 gap.plot(twogrp,gap=c(8,16),xlab="Index",ylab="Group values",
  main="Gap on Y axis",col=gpcol)
 gap.plot(twogrp,rnorm(20),gap=c(8,16),gap.axis="x",xlab="X values",
  xtics=c(4,7,17,20),ylab="Y values",main="Gap on X axis with added lines")
 gap.plot(c(seq(3.5,7.5,by=0.5),seq(16.5,22.5,by=0.5)),
  rnorm(22),gap=c(8,16),gap.axis="x",type="l",add=TRUE,col=2,)
 gap.plot(twogrp,gap=c(8,16,25,35),
 xlab="X values",ylab="Y values",xlim=c(1,30),ylim=c(0,25),
 main="Test two gap plot with the lot",xtics=seq(0,30,by=5),
 ytics=c(4,6,18,20,22,38,40,42),
 lty=c(rep(1,10),rep(2,10)),
 pch=c(rep(2,10),rep(3,10)),
 col=c(rep(2,10),rep(3,10)),
 type="b")
 gap.plot(21:30,rnorm(10)+40,gap=c(8,16,25,35),add=TRUE,
  lty=rep(3,10),col=rep(4,10),type="l")
\end{ExampleCode}
\end{Examples}

