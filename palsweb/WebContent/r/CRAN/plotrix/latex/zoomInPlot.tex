\HeaderA{zoomInPlot}{Display a plot with a rectangular section expanded in an adjacent plot.}{zoomInPlot}
\keyword{misc}{zoomInPlot}
\begin{Description}\relax
Display one plot on the left half of a device and an expanded section of that
plot on the right half of the device with connecting lines showing the expansion.
\end{Description}
\begin{Usage}
\begin{verbatim}
 zoomInPlot(x,y=NULL,xlim=NULL,ylim=NULL,rxlim=xlim,rylim=ylim,xend=NA,
  zoomtitle=NULL,...)
\end{verbatim}
\end{Usage}
\begin{Arguments}
\begin{ldescription}
\item[\code{x,y}] numeric data vectors. If \samp{y} is not specified, it is set equal
to \samp{x} and \samp{x} is set to \samp{1:length(y)}.
\item[\code{xlim,ylim}] Limits for the initial plot.
\item[\code{rxlim,rylim}] Limits for the expanded plot. These must be within the above.
\item[\code{xend}] Where to end the segments that indicate the expansion. Defaults to
just left of the tick labels on the left ordinate.
\item[\code{zoomtitle}] The title of the plot, display in the top center.
\item[\code{...}] additional arguments passed to \samp{plot}.
\end{ldescription}
\end{Arguments}
\begin{Details}\relax
\samp{zoomInPlot} sets up a two column layout in the current device and calls
\samp{plot} to display a plot in the left column. It then draws a rectangle
corresponding to the \samp{rxlim} and \samp{rylim} arguments and displays a
second plot of that rectangle in the right column. It is currently very simple
and will probably become more flexible in future versions.
\end{Details}
\begin{Value}
nil
\end{Value}
\begin{Author}\relax
Jim Lemon
\end{Author}
\begin{SeeAlso}\relax
\samp{\LinkA{plot}{plot}}
\end{SeeAlso}
\begin{Examples}
\begin{ExampleCode}
 zoomInPlot(rnorm(100),rnorm(100),rxlim=c(-1,1),rylim=c(-1,1),
  zoomtitle="Zoom In Plot")
\end{ExampleCode}
\end{Examples}

