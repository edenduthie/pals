\HeaderA{axis.mult}{Display an axis with values having a multiplier}{axis.mult}
\keyword{misc}{axis.mult}
\begin{Description}\relax
An axis is displayed on an existing plot where the tick values are divided
by a multiplier and the multiplier is displayed next to the axis.
\end{Description}
\begin{Usage}
\begin{verbatim}
 axis.mult(side=1,at=NULL,labels,mult=1,mult.label,mult.line,
  mult.labelpos=NULL,...)
\end{verbatim}
\end{Usage}
\begin{Arguments}
\begin{ldescription}
\item[\code{side}] which side to display
\item[\code{at}] where to place the tick marks - defaults to \samp{axTicks()}
\item[\code{labels}] tick labels - defaults to at/mult
\item[\code{mult}] the multiplier factor
\item[\code{mult.label}] the label to show the multiplier - defaults to "x mult"
\item[\code{mult.line}] the margin line upon which to show the multiplier
\item[\code{mult.labelpos}] where to place \samp{mult.label} - defaults to centered
and outside the axis tick labels
\item[\code{...}] additional arguments passed to \samp{axis}.
\end{ldescription}
\end{Arguments}
\begin{Details}\relax
\samp{axis.mult} automates the process of displaying an axis with a 
multiplier applied to the tick values. By default it will divide the
default axis tick labels by \samp{mult} and place \samp{mult.label}
where \samp{xlab} or \samp{ylab} would normally appear. Thus the plot
call should set the relevant label to an empty string in such cases.
It is simplest to call \samp{plot} with \samp{axes=FALSE} and then 
display the box and any standard axes before calling \samp{axis.mult}.
\end{Details}
\begin{Value}
nil
\end{Value}
\begin{Note}\relax
While \samp{axis.mult} will try to display an axis on any side, the top
and right margins will require adjustment using \samp{par} for
\samp{axis.mult} to display properly.
\end{Note}
\begin{Author}\relax
Jim Lemon
\end{Author}
\begin{SeeAlso}\relax
\samp{\LinkA{axis}{axis}}, \samp{\LinkA{mtext}{mtext}}
\end{SeeAlso}
\begin{Examples}
\begin{ExampleCode}
 plot(1:10*0.001,1:10*100,axes=FALSE,xlab="",ylab="",main="Axis multipliers")
 box()
 axis.mult(1,mult=0.001)
 axis.mult(2,mult=100)
\end{ExampleCode}
\end{Examples}

