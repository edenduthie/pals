\HeaderA{piper.diagram}{Display a Piper diagram.}{piper.diagram}
\keyword{misc}{piper.diagram}
\begin{Description}\relax
Displays a Piper diagram.
\end{Description}
\begin{Usage}
\begin{verbatim}
 piper.diagram(ca,mg,so4,cl,ions=data.frame(ca=ca,mg=mg,so4=so4,cl=cl),
  sites=1:NROW(ions),new=FALSE,ppm=TRUE,chull=FALSE,tcsep=0.2,pch=3,
  main="",cex.lab=0.7,cex.tck=0.6,cex.pch=1,grid=TRUE,col=NA,ticklength=0.03,
  lwd.frame=1,pch.lwd=0.7,lwd.grid=lwd.frame,col.box="black",col.tck=col.box,
  col.grid="grey")
\end{verbatim}
\end{Usage}
\begin{Arguments}
\begin{ldescription}
\item[\code{ca,mg,so4,cl}] Concentrations of the ions.
\item[\code{ions}] data frame containing the above values.
\item[\code{sites}] Indices for the ions.
\item[\code{new}] Whether this is a new diagram.
\item[\code{ppm}] Whether the concentrations are in milligrams/liter (parts per
million) or milliequivalents.
\item[\code{chull}] Whether to use the \samp{chull} function.
\item[\code{tcsep}] Spacing for tick mark labels.
\item[\code{pch}] Symbol to use for ion values.
\item[\code{main}] Title for the diagram.
\item[\code{cex.lab}] Expansion for the label text.
\item[\code{cex.tck}] Expansion for the tick text.
\item[\code{cex.pch}] Expansion for the point text.
\item[\code{grid}] Whether to display a grid.
\item[\code{col}] Colors for the points.
\item[\code{ticklength}] Length for the ticks.
\item[\code{lwd.frame}] Line width for the frame.
\item[\code{pch.lwd}] Line width for the symbols.
\item[\code{lwd.grid}] Line width for the grid.
\item[\code{col.box}] Color for the box.
\item[\code{col.tck}] Color for the ticks.
\item[\code{col.grid}] Color for the grid.
\end{ldescription}
\end{Arguments}
\begin{Value}
The return value from piper.points.
\end{Value}
\begin{Author}\relax
Mike Cheetham
\end{Author}

