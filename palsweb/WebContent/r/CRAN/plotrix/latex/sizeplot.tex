\HeaderA{sizeplot}{Plot with repeated symbols by size}{sizeplot}
\keyword{hplot}{sizeplot}
\begin{Description}\relax
Plots a set of (x,y) data with repeated points denoted by larger
symbol sizes
\end{Description}
\begin{Usage}
\begin{verbatim}
 sizeplot(x, y, scale=1, pow=0.5, powscale=TRUE, size=c(1,4), add=FALSE, ...)
\end{verbatim}
\end{Usage}
\begin{Arguments}
\begin{ldescription}
\item[\code{x}] x coordinates of data
\item[\code{y}] y coordinates of data
\item[\code{scale}] scaling factor for size of symbols
\item[\code{pow}] power exponent for size of symbols
\item[\code{powscale}] (logical) use power scaling for symbol size?
\item[\code{size}] (numeric vector) min and max size for scaling, if powscale=FALSE
\item[\code{add}] (logical) add to an existing plot?
\item[\code{...}] other arguments to \samp{plot()} or \samp{points()}
\end{ldescription}
\end{Arguments}
\begin{Details}\relax
Most useful for plotting (e.g.) discrete data, where repeats are
likely.  If all points are repeated equally, gives a warning.  The
size of a point is given by \eqn{scale*n^pow}{}, where n is the number of
repeats, if powscale is TRUE, or it is scaled between size[1] and size[2],
if powscale is FALSE.
\end{Details}
\begin{Value}
A plot is produced on the current device, or points are added to the
current plot if \samp{add=TRUE}.
\end{Value}
\begin{Author}\relax
Ben Bolker
\end{Author}
\begin{SeeAlso}\relax
\samp{\LinkA{symbols}{symbols}}
\end{SeeAlso}
\begin{Examples}
\begin{ExampleCode}
 x <- c(0.1,0.1,0.1,0.1,0.1,0.2,0.2,0.2,0.2,0.3,0.3)
 y <- c( 1,  1,  1,  1,  2,  2,  2,  3,  3,  4,  5 )
 plot(x,y)
 sizeplot(x,y)
 sizeplot(x,y,pch=2)
\end{ExampleCode}
\end{Examples}

