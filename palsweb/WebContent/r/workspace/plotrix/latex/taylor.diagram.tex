\HeaderA{taylor.diagram}{Taylor diagram}{taylor.diagram}
\keyword{misc}{taylor.diagram}
\begin{Description}\relax
Display a Taylor diagram.
\end{Description}
\begin{Usage}
\begin{verbatim}
 taylor.diagram(ref,model,add=FALSE,col="red",pch=19,pos.cor=TRUE,
  xlab="",ylab="",main="Taylor Diagram",show.gamma=TRUE,ngamma=3,
  sd.arcs=0,ref.sd=FALSE,grad.corr.lines=c(0.2,0.4,0.6,0.8,0.9),
  pcex=1,normalize=FALSE,...)
\end{verbatim}
\end{Usage}
\begin{Arguments}
\begin{ldescription}
\item[\code{ref}] numeric vector - the reference values.
\item[\code{model}] numeric vector - the predicted model values.
\item[\code{add}] whether to draw the diagram or just add a point.
\item[\code{col}] the color for the points displayed.
\item[\code{pch}] the type of point to display.
\item[\code{pos.cor}] whether to display only positive (\samp{TRUE}) or all
values of correlation (\samp{FALSE}).
\item[\code{xlab,ylab}] plot axis labels.
\item[\code{main}] title for the plot.
\item[\code{show.gamma}] whether to display standard deviation arcs around
the reference point (only for \samp{pos.cor=TRUE}).
\item[\code{ngamma}] the number of gammas to display (default=3).
\item[\code{sd.arcs}] whether to display arcs along the standard deviation axes
(see Details).
\item[\code{ref.sd}] whether to display the arc representing the reference
standard deviation.
\item[\code{grad.corr.lines}] the values for the radial lines for correlation
values (see Details).
\item[\code{pcex}] character expansion for the plotted points.
\item[\code{normalize}] whether to normalize the models so that the reference
has a standard deviation of 1.
\item[\code{...}] Additional arguments passed to \samp{plot}.
\end{ldescription}
\end{Arguments}
\begin{Details}\relax
The Taylor diagram is used to display the quality of model predictions
against the reference values, typically direct observations.

A diagram is built by plotting one model against the reference,
then adding alternative model points. If \samp{normalize=TRUE} 
when plotting the first model, remember to set it to \samp{TRUE}
when plotting additional models.

Two displays are available. One displays the entire range of correlations
from -1 to 1. Setting \samp{pos.cor} to \samp{FALSE} will produce this
display. The -1 to 1 display includes a radial grid for the correlation
values. When \samp{pos.cor} is set to \samp{TRUE}, only the
range from 0 to 1 will be displayed. The \samp{gamma} lines and the arc at
the reference standard deviation are optional in this display.

Both the standard deviation arcs and the gamma lines are optional in the
\samp{pos.cor=TRUE} version. Setting \samp{sd.arcs} or \samp{grad.corr.lines}
to zero or FALSE will cause them not to be displayed. If more than one value is
passed for \samp{sd.arcs}, the function will try to use the values passed,
otherwise it will call \samp{pretty} to calculate the values.
\end{Details}
\begin{Value}
The values of \samp{par} that preceded the function. This allows the
user to add points to the diagram, then restore the original values. This
is only necessary when using the 0 to 1 correlation range.
\end{Value}
\begin{Author}\relax
Olivier Eterradossi with modifications by Jim Lemon
\end{Author}
\begin{References}\relax
Taylor, K.E. (2001) Summarizing multiple aspects of model performance in a
single diagram. Journal of Geophysical Research, 106: 7183-7192.
\end{References}
\begin{Examples}
\begin{ExampleCode}
 # fake some reference data
 ref<-rnorm(30,sd=2)
 # add a little noise
 model1<-ref+rnorm(30)/2
 # add more noise
 model2<-ref+rnorm(30)
 # display the diagram with the better model
 oldpar<-taylor.diagram(ref,model1)
 # now add the worse model
 taylor.diagram(ref,model2,add=TRUE,col="blue")
 # get approximate legend position
 lpos<-1.5*sd(ref)
 # add a legend
 legend(lpos,lpos,legend=c("Better","Worse"),pch=19,col=c("red","blue"))
 # now restore par values
 par(oldpar)
 # show the "all correlation" display
 taylor.diagram(ref,model1,pos.cor=FALSE)
 taylor.diagram(ref,model2,add=TRUE,col="blue")
\end{ExampleCode}
\end{Examples}

