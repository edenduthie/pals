\HeaderA{plot.dendrite}{Plot a dendrogram of a dendrite object.}{plot.dendrite}
\keyword{misc}{plot.dendrite}
\begin{Description}\relax
Plot a dendrogram for two or more mutually exclusive attributes.
\end{Description}
\begin{Usage}
\begin{verbatim}
 ## S3 method for class 'dendrite':
 plot(x,xlabels=NULL,main="",mar=c(1,0,3,0),cex=1,...)
\end{verbatim}
\end{Usage}
\begin{Arguments}
\begin{ldescription}
\item[\code{x}] A \samp{dendrite} object containing the counts of objects having
combinations of mutually exclusive attributes.
\item[\code{xlabels}] The category labels that will be displayed beneath the
dendrogram.
\item[\code{main}] The title of the plot.
\item[\code{mar}] Margins for the plot.
\item[\code{cex}] Character expansion for the leaves of the dendrogram.
\item[\code{...}] Additional arguments passed to \samp{plot}.
\end{ldescription}
\end{Arguments}
\begin{Details}\relax
\samp{plot.dendrite} sets up a plot for a dendrogram. The actual plotting of
the dendrogram is done by \samp{furc}.
\end{Details}
\begin{Value}
nil
\end{Value}
\begin{Author}\relax
Jim Lemon
\end{Author}
\begin{SeeAlso}\relax
\samp{\LinkA{furc}{furc}}
\end{SeeAlso}
\begin{Examples}
\begin{ExampleCode}
 sex<-sample(c("M","F"),100,TRUE)
 hair<-sample(c("Blond","Black","Brown","Red"),100,TRUE)
 eye<-sample(c("Blue","Black","Brown","Green"),100,TRUE)
 charac<-data.frame(sex=sex,hair=hair,eye=eye)
 characlist<-makeDendrite(charac)
 plot.dendrite(characlist,names(charac),main="Test dendrogram",cex=0.8)
\end{ExampleCode}
\end{Examples}

