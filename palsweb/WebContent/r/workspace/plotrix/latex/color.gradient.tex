\HeaderA{color.gradient}{Calculate an arbitrary sequence of colors.}{color.gradient}
\keyword{misc}{color.gradient}
\begin{Description}\relax
\samp{color.gradient} is now just a call to \samp{color.scale} with a
vector of equally spaced integers (1:nslices). The function is kept for
backward compatibility.
\end{Description}
\begin{Usage}
\begin{verbatim}
 color.gradient(reds,greens,blues,nslices=50)
\end{verbatim}
\end{Usage}
\begin{Arguments}
\begin{ldescription}
\item[\code{reds,greens,blues}] vectors of the values of the color components
as 0 to 1.
\item[\code{nslices}] The number of color "slices".
\end{ldescription}
\end{Arguments}
\begin{Value}
A vector of hexadecimal color values as used by \samp{col}.
\end{Value}
\begin{Note}\relax
The function is mainly useful for defining a set of colors to represent
a known number of gradations. Such a set can be used to assign a grade
to a small number of values (e.g. points on a scatterplot - but see 
\samp{color.scale} for large numbers) and display a color bar using
\samp{gradient.rect} as a legend.
\end{Note}
\begin{Author}\relax
Jim Lemon
\end{Author}
\begin{SeeAlso}\relax
\samp{\LinkA{rescale}{rescale}},\samp{\LinkA{approx}{approx}},\samp{\LinkA{color.scale}{color.scale}}
\end{SeeAlso}
\begin{Examples}
\begin{ExampleCode}
 # try it with red and blue endpoints and green midpoints.
 color.gradient(c(0,1),c(1,0.6,0.4,0.3,0),c(0.1,0.6))
\end{ExampleCode}
\end{Examples}

