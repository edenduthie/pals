\HeaderA{lengthKey}{Key for interpreting lengths in a plot}{lengthKey}
\keyword{misc}{lengthKey}
\begin{Description}\relax
Key for interpreting lengths in a plot
\end{Description}
\begin{Usage}
\begin{verbatim}
 lengthKey(x,y,tickpos,scale)
\end{verbatim}
\end{Usage}
\begin{Arguments}
\begin{ldescription}
\item[\code{x,y}] The position of the left end of the key in user units.
\item[\code{tickpos}] The labels that will appear above the key.
\item[\code{scale}] A value that will scale the length of the key.
\end{ldescription}
\end{Arguments}
\begin{Details}\relax
\samp{lengthKey} displays a line with tick marks and the values in
\samp{tickpos} above those tickmarks. It is useful when line segments
on a plot represent numeric values. Note that if the plot does not have
a 1:1 aspect ratio, a length key is usually misleading.
\end{Details}
\begin{Value}
nil
\end{Value}
\begin{Author}\relax
Jim Lemon
\end{Author}
\begin{SeeAlso}\relax
\samp{\LinkA{segments}{segments}}, \samp{\LinkA{arrows}{arrows}}
\end{SeeAlso}
\begin{Examples}
\begin{ExampleCode}
 # manufacture a matrix of orientations in radians
 o<-matrix(rep(pi*seq(0.1,0.8,by=0.1),7),ncol=8,byrow=TRUE)
 m<-matrix(rnorm(56)+4,ncol=8,byrow=TRUE)
 # get an empty plot of approximately 1:1 aspect ratio
 plot(0,xlim=c(0.7,8.3),ylim=c(0.7,7.3),type="n")
 vectorField(o,m,vecspec="rad")
 # the scaling usually has to be worked out by trial and error
 lengthKey(0.3,-0.5,c(0,5,10),0.24)
\end{ExampleCode}
\end{Examples}

