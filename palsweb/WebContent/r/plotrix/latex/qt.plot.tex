\HeaderA{qt.plot}{Quantity by interval plot}{qt.plot}
\keyword{misc}{qt.plot}
\begin{Description}\relax
Display the counts of numeric quantities against the intervals between 
quantities that are equal when rounded to integers.
\end{Description}
\begin{Usage}
\begin{verbatim}
 qt.plot(qnt,qtime=NA,col=NULL,border="lightgray",
  main="Quantity x interval",xlab="Interval",ylab="Quantity",
  mar=c(5,4,4,4),...)
\end{verbatim}
\end{Usage}
\begin{Arguments}
\begin{ldescription}
\item[\code{qnt}] Numeric vector
\item[\code{qtime}] Numeric vector - may be a date as an integer.
\item[\code{col}] The colors to fill the strips. NA for none.
\item[\code{border}] border color for the polygons
\item[\code{main}] The title of the plot.
\item[\code{xlab,ylab}] Axis labels.
\item[\code{mar}] margins for the plot - defaults to leave space for scale
\item[\code{...}] additional arguments passed to \samp{plot}.
\end{ldescription}
\end{Arguments}
\begin{Details}\relax
The intervals calculated from \samp{qtime} are the x values and the counts
of values of \samp{qnt} are the y values of the plot displayed as the widths
of sections of polygons running across the time intervals. This plot was
devised to display the distribution of drinking, but may be useful for any
situation in which it is desired to display the distribution of numerically
coded quantities against the intervals between their occurrence. Note that if
there are many values and many intervals, the resulting plot will be mostly
empty. Categorizing the values and intervals so that there are only three or
four categories will often produce a more informative plot.

\samp{qt.plot} assumes that the values in \samp{qtime} represent interpretable
intervals like seconds or days. The default is to assume sequential time
intervals. If \samp{qtime} contains dates, they must be translated to
numeric format. These values will be sorted by the function. If \samp{qtime} is NA,
it will be assigned \samp{1:length(qnt)}.
\end{Details}
\begin{Value}
nil
\end{Value}
\begin{Author}\relax
Jim Lemon
\end{Author}
\begin{SeeAlso}\relax
\samp{\LinkA{polygon}{polygon}}
\end{SeeAlso}
\begin{Examples}
\begin{ExampleCode}
 # first a moderate drinker with frequent bigger sessions
 qnt<-sample(0:5,365,TRUE,prob=c(0.02,0.1,0.4,0.3,0.1,0.08))
 qtdates<-seq(as.Date("2007-01-01"),as.Date("2007-12-31"),by=1)
 qt.plot(qnt,as.numeric(qtdates),xlab="Number of days interval",
  ylab="Standard drinks per session")
 # now add monthly bigger sessions and notice how this
 qnt[c(30,60,90,120,150,180,210,240,270,300,330,360)]<-rep(4:5,length.out=12)
 qt.plot(qnt,as.numeric(qtdates),xlab="Number of days interval",
  ylab="Standard drinks per session")
\end{ExampleCode}
\end{Examples}

