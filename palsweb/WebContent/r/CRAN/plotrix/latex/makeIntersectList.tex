\HeaderA{makeIntersectList}{Count set intersections}{makeIntersectList}
\keyword{misc}{makeIntersectList}
\begin{Description}\relax
Create a list of set intersections from a matrix of indicators.
\end{Description}
\begin{Usage}
\begin{verbatim}
 makeIntersectList(x,xnames=NULL)
\end{verbatim}
\end{Usage}
\begin{Arguments}
\begin{ldescription}
\item[\code{x}] A data frame or matrix where rows represent objects and columns
attributes. A \samp{1} or \samp{TRUE} indicates that the object (row) has
that attribute or is a member of that set (column).
\item[\code{xnames}] Optional user-supplied names for the categories of x.
\end{ldescription}
\end{Arguments}
\begin{Details}\relax
\samp{makeIntersectList} reads a matrix (or data frame) containing only
dichotomous values (either 0/1 or FALSE/TRUE). Each row represents an object
and each column represents a set. A value of 1 or TRUE indicates that that
object is a member of that set. The function creates a list of vectors that
correspond to all combinations of the sets (set intersections) and inserts the
counts of elements in each combination. If a row of the matrix is all zeros,
it will not be counted, but the last element of the list returned contains the
count of rows in \samp{x} and thus non-members can be calculated.

makeIntersectList combines the set or attribute names to form
intersection names. For the intersection of sets A and B, the name will
be AB and so on. These are the names that will be displayed by
intersectDiagram. To change these, use the \samp{xnames} argument.
\end{Details}
\begin{Value}
A list of the intersection counts or percentages and the total number of
objects.
\end{Value}
\begin{Author}\relax
Jim Lemon
\end{Author}
\begin{SeeAlso}\relax
\samp{\LinkA{intersectDiagram}{intersectDiagram}}, \samp{\LinkA{pasteCols}{pasteCols}}
\end{SeeAlso}
\begin{Examples}
\begin{ExampleCode}
 # create a matrix where each row represents an element and
 # a 1 (or TRUE) in each column indicates that the element is a member
 # of that set.
 setdf<-data.frame(A=sample(c(0,1),100,TRUE,prob=c(0.7,0.3)),
  B=sample(c(0,1),100,TRUE,prob=c(0.7,0.3)),
  C=sample(c(0,1),100,TRUE,prob=c(0.7,0.3)),
  D=sample(c(0,1),100,TRUE,prob=c(0.7,0.3)))
 setdflist<-makeIntersectList(setdf)
 setdflist
\end{ExampleCode}
\end{Examples}

