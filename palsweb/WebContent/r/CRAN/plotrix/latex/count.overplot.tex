\HeaderA{count.overplot}{Show overlying points as counts}{count.overplot}
\keyword{misc}{count.overplot}
\begin{Description}\relax
\samp{count.overplot} checks for overlying points defined as points
separated by a maximum of \samp{tol}, a two element numeric vector of
the x and y tolerance.  Defaults to 1/2 of the width of a lower case 
"o" in the x direction and 1/2 of the height of a lower case "o" in 
the y direction.
\end{Description}
\begin{Usage}
\begin{verbatim}
 count.overplot(x,y,tol=NULL,col=par("fg"),pch="1",...)
\end{verbatim}
\end{Usage}
\begin{Arguments}
\begin{ldescription}
\item[\code{x,y}] Two numeric data vectors or the first two columns of a matrix
or data frame. Typically the x/y coordinates of points to be plotted.
\item[\code{tol}] The largest distance between points that will be considered
to be overlying.
\item[\code{col}] Color(s) for the points (not the numbers).
\item[\code{pch}] Symbol(s) to display.
\item[\code{...}] additional arguments passed to \samp{plot}.
\end{ldescription}
\end{Arguments}
\begin{Value}
nil
\end{Value}
\begin{Author}\relax
Jim Lemon
\end{Author}
\begin{SeeAlso}\relax
\samp{\LinkA{cluster.overplot}{cluster.overplot}},\samp{\LinkA{sizeplot}{sizeplot}}
\end{SeeAlso}
\begin{Examples}
\begin{ExampleCode}
 xy.mat<-cbind(sample(1:10,200,TRUE),sample(1:10,200,TRUE))
 count.overplot(xy.mat,main="count.overplot",
  xlab="X values",ylab="Y values")
\end{ExampleCode}
\end{Examples}

