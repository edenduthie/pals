\HeaderA{gradient.rect}{Display a rectangle filled with an arbitrary color gradient.}{gradient.rect}
\keyword{misc}{gradient.rect}
\begin{Description}\relax
\samp{gradient.rect} draws a rectangle consisting of \samp{nslices}
subrectangles of the colors in \samp{col} or those returned by 
\samp{color.gradient} if \samp{col} is NULL. The rectangle is
'sliced' in the direction specified by \samp{gradient}.
\end{Description}
\begin{Usage}
\begin{verbatim}
 gradient.rect(xleft,ybottom,xright,ytop,reds,greens,blues,col=NULL,
  nslices=50,gradient="x",border=par("fg"))
\end{verbatim}
\end{Usage}
\begin{Arguments}
\begin{ldescription}
\item[\code{xleft,ybottom,xright,ytop}] Positions of the relevant corners
of the desired rectangle, as in \samp{rect}.
\item[\code{reds,greens,blues}] vectors of the values of the color components
either as 0 to 1 or ,if any value is greater than 1, 0 to 255.
\item[\code{col}] Vector of colors. If supplied, this takes precedence over
\samp{reds, greens, blues} and \samp{nslices} will be set to its length.
\item[\code{nslices}] The number of sub-rectangles that will be drawn.
\item[\code{gradient}] whether the gradient should be horizontal (x) or vertical.
\item[\code{border}] The color of the border around the rectangle (NA for none).
\end{ldescription}
\end{Arguments}
\begin{Value}
the vector of hexadecimal color values from \samp{color.gradient} or
\samp{col}.
\end{Value}
\begin{Author}\relax
Jim Lemon
\end{Author}
\begin{Examples}
\begin{ExampleCode}
 # get an empty box
 plot(0:10,type="n",axes=FALSE)
 # run across the three primaries
 gradient.rect(1,0,3,6,reds=c(1,0),
  greens=c(seq(0,1,length=10),seq(1,0,length=10)),
  blues=c(0,1),gradient="y")
 # now a "danger gradient"
 gradient.rect(4,0,6,6,c(seq(0,1,length=10),rep(1,10)),
  c(rep(1,10),seq(1,0,length=10)),c(0,0),gradient="y")
 # now just a smooth gradient across the bar
 gradient.rect(7,0,9,6,col=smoothColors("red",38,"blue"),border=NA)
\end{ExampleCode}
\end{Examples}

