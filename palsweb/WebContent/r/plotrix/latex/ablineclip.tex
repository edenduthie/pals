\HeaderA{ablineclip}{Add a straight line to a plot}{ablineclip}
\keyword{aplot}{ablineclip}
\begin{Description}\relax
As \samp{abline}, but has arguments \samp{x1,x2,y1,y2} as in \samp{clip}.
\end{Description}
\begin{Usage}
\begin{verbatim}
 ablineclip(a=NULL,b=NULL,h=NULL,v=NULL,reg=NULL,coef=NULL,untf=FALSE,
  x1=NULL,x2=NULL,y1=NULL,y2=NULL,...)
\end{verbatim}
\end{Usage}
\begin{Arguments}
\begin{ldescription}
\item[\code{a}] Intercept.
\item[\code{b}] Slope.
\item[\code{h}] the x-value(s) for vertical line(s).
\item[\code{v}] the y-value(s) for horizontal line(s).
\item[\code{reg}] Fitted lm object. 
\item[\code{coef}] Coefficients, typically intercept and slope.
\item[\code{untf}] How to plot on log coordinates, see \samp{abline}.
\item[\code{x1,x2,y1,y2}] Clipping limits, see \samp{clip}.
\item[\code{...}] Further arguments passed to \samp{abline}.
\end{ldescription}
\end{Arguments}
\begin{Details}\relax
\samp{ablineclip} sets a new clipping region and then calls \samp{abline}.
If any of the four clipping limits is NULL, the values from \samp{par("usr")}
are substituted. After the call to \samp{abline}, the old clipping region
is restored. In order to make \samp{clip} work, there is a call to \samp{abline}
that draws a line off the plot.
\end{Details}
\begin{Value}
None. Adds to the current plot.
\end{Value}
\begin{Author}\relax
Remko Duursma
\end{Author}
\begin{SeeAlso}\relax
\samp{\LinkA{abline}{abline}}
\end{SeeAlso}
\begin{Examples}
\begin{ExampleCode}
 x <- rnorm(100)
 y <- x + rnorm(100)
 lmfit <- lm(y~x)
 plot(x,y,xlim=c(-3.5,3.5))
 ablineclip(lmfit,x1=-2,x2=2,lty=2)
 ablineclip(h=0,x1=-2,x2=2,lty=3,col="red")
 ablineclip(v=0,y1=-2.5,y2=1.5,lty=4,col="green")
\end{ExampleCode}
\end{Examples}

