\HeaderA{polar.plot}{Plot values on a circular grid of 0 to 360 degrees.}{polar.plot}
\keyword{misc}{polar.plot}
\begin{Description}\relax
\samp{polar.plot} displays a plot of radial lines, symbols or a polygon 
centered at the midpoint of the plot frame on a 0:360 circle.
Positions are interpreted as beginning at the right and moving
counterclockwise unless \samp{start} specifies another starting point or
\samp{clockwise} is TRUE.
\end{Description}
\begin{Usage}
\begin{verbatim}
 polar.plot(lengths,polar.pos=NULL,labels,label.pos=NULL,
  start=0,clockwise=FALSE,rp.type="r",...)
\end{verbatim}
\end{Usage}
\begin{Arguments}
\begin{ldescription}
\item[\code{lengths}] numeric data vector. Magnitudes will be represented as
the radial positions of symbols, line ends or polygon vertices.
\item[\code{polar.pos}] numeric vector of positions on a 0:360 degree circle.
These will be converted to radians when passed to \samp{radial.plot}.
\item[\code{labels}] text labels to place on the periphery of the circle. This 
defaults to labels every 20 degrees. For no labels, pass an empty string.
\item[\code{label.pos}] positions of the peripheral labels in degrees
\item[\code{start}] The position for zero degrees on the plot in degrees.
\item[\code{clockwise}] Whether to increase angles clockwise rather than the
default counterclockwise.
\item[\code{rp.type}] Whether to plot radial lines, symbols or a polygon.
\item[\code{...}] additional arguments passed to \samp{radial.plot} and
then to \samp{plot}.
\end{ldescription}
\end{Arguments}
\begin{Value}
nil
\end{Value}
\begin{Author}\relax
Jim Lemon
\end{Author}
\begin{SeeAlso}\relax
\samp{\LinkA{radial.plot}{radial.plot}}
\end{SeeAlso}
\begin{Examples}
\begin{ExampleCode}
 testlen<-c(rnorm(36)*2+5)
 testpos<-seq(0,350,by=10)
 polar.plot(testlen,testpos,main="Test Polar Plot",lwd=3,line.col=4)
 polar.plot(testlen,testpos,main="Test Clockwise Polar Plot",
  start=90,clockwise=TRUE,lwd=3,line.col=4)
 # reset the margins
 par(mar=c(5,4,4,2))
\end{ExampleCode}
\end{Examples}

