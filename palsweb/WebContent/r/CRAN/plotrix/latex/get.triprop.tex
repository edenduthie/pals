\HeaderA{get.triprop}{Enter three proportion data - usually soil textures.}{get.triprop}
\keyword{misc}{get.triprop}
\begin{Description}\relax
\samp{get.triprop} allows the user to enter triplets of proportions
or percentages of three components such as sand, silt and clay in soils.
\end{Description}
\begin{Usage}
\begin{verbatim}
 get.triprop(use.percentages=FALSE,cnames=c("1st","2nd","3rd"))
\end{verbatim}
\end{Usage}
\begin{Arguments}
\begin{ldescription}
\item[\code{use.percentages}] Logical - whether to treat the entries as
percentages and scale to proportions.
\item[\code{cnames}] column names for the resulting three column matrix.
\end{ldescription}
\end{Arguments}
\begin{Details}\relax
The three proportions of each row must sum to 100 or 1 within 1\% or
the function will warn the operator.
\end{Details}
\begin{Value}
A matrix of the components of one or more observations.
\end{Value}
\begin{Author}\relax
Jim Lemon
\end{Author}
\begin{SeeAlso}\relax
\samp{\LinkA{triax.plot}{triax.plot}}, \samp{\LinkA{soil.texture}{soil.texture}}
\end{SeeAlso}
\begin{Examples}
\begin{ExampleCode}
 if(dev.interactive()) {
  # get some proportions
  newsp<-get.triprop()
  # show the triangle
  triax.frame(main="Test triax.plot")
  # now plot the observations
  triax.points(newsp)
 }
\end{ExampleCode}
\end{Examples}

