\HeaderA{polygon.shadow}{Display a shadow effect for an arbitrary polygon}{polygon.shadow}
\keyword{misc}{polygon.shadow}
\begin{Description}\relax
Displays a shadow effect on an existing plot
\end{Description}
\begin{Usage}
\begin{verbatim}
 polygon.shadow(x,y=NULL,offset=NA,inflate=NA,col=c("#ffffff","#cccccc"))
\end{verbatim}
\end{Usage}
\begin{Arguments}
\begin{ldescription}
\item[\code{x,y}] x and y coordinate of the vertices of the polygon. \samp{y}
can be missing if \samp{x} is a list with \samp{x} and \samp{y} components.
\item[\code{offset}] a vector containing the values of the x and y offsets for the
shadow. Defaults to 1/20 of the maximum x and y dimensions of the polygon.
\item[\code{col}] the colors of the shadow from the outer edge to the central part.
\item[\code{inflate}] the amount to "inflate" the shadow relative to the polygon 
(i.e. the penumbra). Defaults to the values in \samp{offset}.
\end{ldescription}
\end{Arguments}
\begin{Details}\relax
\samp{polygon.shadow} is typically called just before drawing a polygon.
It displays a shadow effect by drawing the polygon ten times, beginning 
with the first color in \samp{col} and stepping through to the second
color to create a "shadow" (or a "halo" if you prefer). Each successive
polygon is shrunk by 10\% of \samp{inflate}. The default shadow effect has
the light at the upper left. This effect may also be used as a text
background.
\end{Details}
\begin{Value}
nil
\end{Value}
\begin{Note}\relax
The background must be a constant color or the shadow effect will not 
look right.
\end{Note}
\begin{Author}\relax
Jim Lemon
\end{Author}
\begin{SeeAlso}\relax
\samp{\LinkA{polygon}{polygon}}
\end{SeeAlso}
\begin{Examples}
\begin{ExampleCode}
 par(pty="s")
 plot(1:5,type="n",main="Polygon Shadow test",xlab="",ylab="",axes=FALSE)
 box()
 # do a shadow on a yellow square
 polygon(c(1,2.2,2.2,1),c(5,5,3.8,3.8),col="#ffff00")
 polygon.shadow(c(1.2,2,2,1.2),c(4.8,4.8,4,4),col=c("#ffff00","#cccc00"))
 polygon(c(1.2,2,2,1.2),c(4.8,4.8,4,4),col=c("#ff0000"))
 # a green triangle on a light blue square with a big offset
 polygon(c(4,5,5,4),c(2,2,1,1),col="#aaaaff")
 polygon.shadow(c(4.5,4.8,4.2),c(1.7,1.2,1.2),col=c("#aaaaff","#8888cc"),
  offset=c(0.1,-0.1),inflate=c(0.2,0.2))
 polygon(c(4.5,4.8,4.2),c(1.7,1.2,1.2),col=c("#00ff00"))
 # now a circle as a background
 polygon.shadow(cos(seq(0,2*pi,by=pi/20))+3,sin(seq(0,2*pi,by=pi/20))+3,
  offset=c(0,0),inflate=c(0.1,0.1))
 text(3,3,"Polygon shadow\nas a circular\ntext background",cex=1.5)
\end{ExampleCode}
\end{Examples}

