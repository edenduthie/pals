\HeaderA{getFigCtr}{Find the center of the figure region.}{getFigCtr}
\keyword{misc}{getFigCtr}
\begin{Description}\relax
Calculates the center of the figure region.
\end{Description}
\begin{Usage}
\begin{verbatim}
 getFigCtr()
\end{verbatim}
\end{Usage}
\begin{Details}\relax
\samp{getFigCtr} reads parameters about the current plot and calculates the
vertical and horizontal centers of the figure region. This is typically useful
for placing a centered title on plots where the left and right margins are very
different.
\end{Details}
\begin{Value}
A two element vector containing the coordinates of the center of the
figure region in user units.
\end{Value}
\begin{Author}\relax
Jim Lemon
\end{Author}

