\HeaderA{makeDendrite}{Build a list of the mutually exclusive attributes of objects.}{makeDendrite}
\keyword{misc}{makeDendrite}
\begin{Description}\relax
Build a list of mutually exclusive attributes from a matrix of 
category indicators.
\end{Description}
\begin{Usage}
\begin{verbatim}
 makeDendrite(x)
\end{verbatim}
\end{Usage}
\begin{Arguments}
\begin{ldescription}
\item[\code{x}] A data frame or matrix where rows represent objects and columns
mutually exclusive attributes of a given class.
\end{ldescription}
\end{Arguments}
\begin{Details}\relax
The values in \samp{x} indicate which attribute of a particular class is
possessed by the object. For instance, the attributes dead and alive are
mutually exclusive. \samp{makeDendrite} creates a nested list that contains
the counts of successive combinations of the attributes. The top level
attributes are taken from the first column, then those are combined with
the attributes in the second column and so on.
\end{Details}
\begin{Value}
A list of the counts of objects for each combination of the attribute
classes.
\end{Value}
\begin{Author}\relax
Jim Lemon
\end{Author}
\begin{SeeAlso}\relax
\samp{\LinkA{plot.dendrite}{plot.dendrite}}
\end{SeeAlso}
\begin{Examples}
\begin{ExampleCode}
 sex<-sample(c("M","F"),100,TRUE)
 hair<-sample(c("Blond","Black","Brown","Red"),100,TRUE)
 eye<-sample(c("Blue","Black","Brown","Green"),100,TRUE)
 charac<-data.frame(sex=sex,hair=hair,eye=eye)
 characlist<-makeDendrite(charac)
 characlist
\end{ExampleCode}
\end{Examples}

