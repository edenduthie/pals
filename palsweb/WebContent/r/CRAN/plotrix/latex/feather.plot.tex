\HeaderA{feather.plot}{Display vectors along a horizontal reference line.}{feather.plot}
\keyword{misc}{feather.plot}
\begin{Description}\relax
Displays vectors along a line usually representing time or position.
\end{Description}
\begin{Usage}
\begin{verbatim}
 feather.plot(r,theta,xpos,yref=0,use.arrows=TRUE,
 col.refline="lightgray",fp.type="s",main="",xlab="",ylab="",
 xlabels=NULL,...)
\end{verbatim}
\end{Usage}
\begin{Arguments}
\begin{ldescription}
\item[\code{r}] radii of vectors
\item[\code{theta}] direction of vectors in radians
\item[\code{xpos}] where to start each vector along the reference line
\item[\code{yref}] vertical position to place the reference line
\item[\code{use.arrows}] whether to put arrow heads on the ends of the vectors
\item[\code{col.refline}] the color of the reference line
\item[\code{fp.type}] whether to use "standard" coordinates (begin at the
right and move counterclockwise) or "meteorological" coordinates
(begin at the top and move clockwise) when interpreting the values
of \samp{theta}
\item[\code{main}] the title of the plot
\item[\code{xlab}] the label for the reference line
\item[\code{ylab}] the label for the vertical axis
\item[\code{xlabels}] optional labels for the reference line
\item[\code{...}] additional arguments passed to \samp{arrows} or \samp{segments}
\end{ldescription}
\end{Arguments}
\begin{Details}\relax
This function places vectors of length \samp{r} and angle \samp{theta} 
along a reference line that may represent time or position or some other
value. The user is responsible for spacing the vectors so that they do
not overlap if this is desired.
\end{Details}
\begin{Value}
nil
\end{Value}
\begin{Author}\relax
Jim Lemon, Eduardo Klein
\end{Author}
\begin{SeeAlso}\relax
\samp{\LinkA{spread.labels}{spread.labels}}
\end{SeeAlso}
\begin{Examples}
\begin{ExampleCode}
 feather.plot(0.6+rnorm(8)/5,seq(0,7*pi/4,by=pi/4),1:8,
  main="Standard Coordinates",xlab="Time",ylab="Value")
 if(dev.interactive()) par(ask=TRUE)
 feather.plot(0.6+rnorm(8)/5,seq(0,7*pi/4,by=pi/4),1:8,
  main="Meteorological Coordinates",xlab="Time",ylab="Value",
  fp.type="m",xlabels=TRUE)
 par(ask=FALSE)
\end{ExampleCode}
\end{Examples}

