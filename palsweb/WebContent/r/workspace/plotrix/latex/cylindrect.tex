\HeaderA{cylindrect}{Display an apparent cylinder}{cylindrect}
\keyword{misc}{cylindrect}
\begin{Description}\relax
Display rectangles shaded to appear like cylinders.
\end{Description}
\begin{Usage}
\begin{verbatim}
 cylindrect(xleft,ybottom,xright,ytop,col,border=NA,gradient="x",nslices=50)
\end{verbatim}
\end{Usage}
\begin{Arguments}
\begin{ldescription}
\item[\code{xleft}] The position of the left side of the rectangle(s).
\item[\code{ybottom}] The position of the bottom of the rectangle(s).
\item[\code{xright}] The position of the right side of the rectangle(s).
\item[\code{ytop}] The position of the top side of the rectangle(s).
\item[\code{col}] The base color(s) of the rectangles.
\item[\code{border}] Whether to draw a border and what color.
\item[\code{gradient}] Whether to vary the shading horizontally ("x" - the default)
or vertically (anything but "x").
\item[\code{nslices}] The number of "slices" of color for shading.
\end{ldescription}
\end{Arguments}
\begin{Details}\relax
\samp{cylindrect} displays a rectangle filled with "slices" of color that
simulate the appearance of a cylinder. The slices are calculated so that the 
base color appears at the right or bottom edge of the rectangle, become
progressively lighter to a "highlight" at two thirds of the width or height
and then darken toward the base color again.

The appearance is of a cylinder lit from above and to the left of the viewer.
The position of the apparent light source is hard coded into the function.
\end{Details}
\begin{Value}
The base color(s) of the rectangle(s).
\end{Value}
\begin{Author}\relax
Jim Lemon
\end{Author}
\begin{SeeAlso}\relax
\samp{\LinkA{gradient.rect}{gradient.rect}}
\end{SeeAlso}
\begin{Examples}
\begin{ExampleCode}
 plot(0,xlim=c(0,5),ylim=c(0,5),main="Examples of pseudocylindrical rectangles",
  xlab="",ylab="",axes=FALSE,type="n")
 cylindrect(0,0,1,5,"red")
 cylindrect(rep(1,3),c(0,2,4),rep(4,3),c(1,3,5),"green",gradient="y")
 cylindrect(4,0,5,5,"#8844aa")
\end{ExampleCode}
\end{Examples}

