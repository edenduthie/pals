\HeaderA{get.gantt.info}{Gather the information to create a Gantt chart}{get.gantt.info}
\keyword{misc}{get.gantt.info}
\begin{Description}\relax
Allows the user to enter the information for a Gantt chart.
\end{Description}
\begin{Usage}
\begin{verbatim}
 get.gantt.info(format="%Y/%m/%d")
\end{verbatim}
\end{Usage}
\begin{Arguments}
\begin{ldescription}
\item[\code{format}] the format to be used in entering dates/times. Defaults to
YYYY/mm/dd. See \samp{\LinkA{strptime}{strptime}} for various date/time formats.
\end{ldescription}
\end{Arguments}
\begin{Value}
The list used to create the chart. Elements are:
\begin{ldescription}
\item[\code{labels}] The task labels to be displayed at the left of the chart.
\item[\code{starts,ends}] The task starts/ends as POSIXct dates/times.
\item[\code{priorities}] Task priorities as integers in the range 1 to 10.
There can be less than 10 levels of priority, but if priorities do
not start at 1 (assumed to be the highest), the default priority colors
will be calculated from 1.
\end{ldescription}
\end{Value}
\begin{Author}\relax
Jim Lemon
\end{Author}
\begin{SeeAlso}\relax
\samp{\LinkA{gantt.chart}{gantt.chart}}
\end{SeeAlso}
\begin{Examples}
\begin{ExampleCode}
 cat("Enter task times using HH:MM (hour:minute) format\n")
 get.gantt.info("%H:%M")
\end{ExampleCode}
\end{Examples}

