\HeaderA{smoothColors}{Build a vector of color values.}{smoothColors}
\keyword{misc}{smoothColors}
\begin{Description}\relax
\samp{smoothColors} calculates a sequence of colors. If two color names 
in the arguments are separated by a number, that number of interpolated
colors will be inserted between the two color endpoints. Any number of
color names and integers may be passed, but the last argument must be
a color name. If more than one integer appears between two color names,
only the first will be used in the interpolation and the others will be
ignored.
\end{Description}
\begin{Usage}
\begin{verbatim}
 smoothColors(...,alpha=NA)
\end{verbatim}
\end{Usage}
\begin{Arguments}
\begin{ldescription}
\item[\code{...}] an arbitrary sequence of color names and integers beginning
and ending with a color name.
\item[\code{alpha}] optional \samp{alpha} (transparency) value.
\end{ldescription}
\end{Arguments}
\begin{Value}
A vector of hexadecimal color values as used by \samp{col}.
\end{Value}
\begin{Author}\relax
Barry Rowlingson
\end{Author}
\begin{SeeAlso}\relax
\samp{\LinkA{color.gradient}{color.gradient}},\samp{\LinkA{rgb}{rgb}}
\end{SeeAlso}
\begin{Examples}
\begin{ExampleCode}
 plot(1:10,main="Test opaque colors",type="n",axes=FALSE)
 box()
 rect(1:7,1:7,3:9,3:9,col=smoothColors("red",2,"green",2,"blue"))
\end{ExampleCode}
\end{Examples}

