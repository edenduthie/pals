\HeaderA{triax.points}{Triangle plot points}{triax.points}
\keyword{misc}{triax.points}
\begin{Description}\relax
Display points on a triangle plot.
\end{Description}
\begin{Usage}
\begin{verbatim}
 triax.points(x,show.legend=FALSE,label.points=FALSE,point.labels=NULL,
 col.symbols=par("fg"),pch=par("pch"),bg.symbols=par("bg"),cc.axes=FALSE,...)
\end{verbatim}
\end{Usage}
\begin{Arguments}
\begin{ldescription}
\item[\code{x}] Matrix or data frame where each row is three proportions or 
percentages that must sum to 1 or 100 respectively.
\item[\code{show.legend}] Logical - whether to display a legend.
\item[\code{label.points}] Logical - whether to call \samp{thigmophobe.labels}
to label the points.
\item[\code{point.labels}] Optional labels for the points and/or legend.
\item[\code{col.symbols}] Color of the symbols representing each value.
\item[\code{pch}] Symbols to use in plotting values.
\item[\code{bg.symbols}] Background color for plotting symbols.
\item[\code{cc.axes}] Clockwise or counterclockwise axes and ticks.
\item[\code{...}] Additional arguments passed to \samp{points}.
\end{ldescription}
\end{Arguments}
\begin{Details}\relax
In order for \samp{triax.points} to add points to an existing plot,
the argument \samp{no.add} in the initial call to \samp{triax.plot} 
must be set to \samp{FALSE}. Failing to do this will result in the
points being plotted in the wrong places. It is then up to the user
to call \samp{par} as in the example below to restore plotting
parameters altered during the triangle plot.

\samp{triax.points} displays each triplet of proportions or percentages
as a symbol on the triangle plot. Unless each triplet sums to 1 
(or 100), they will not plot properly and \samp{triax.points} will 
complain appropriately.
\end{Details}
\begin{Value}
A list of the \samp{x,y} positions plotted.
\end{Value}
\begin{Author}\relax
Jim Lemon
\end{Author}
\begin{SeeAlso}\relax
\samp{\LinkA{triax.plot}{triax.plot}},\samp{\LinkA{thigmophobe.labels}{thigmophobe.labels}}
\end{SeeAlso}
\begin{Examples}
\begin{ExampleCode}
 data(soils)
 triax.return<-triax.plot(soils[1:10,],
  main="Adding points to a triangle plot",no.add=FALSE)
 triax.points(soils[11:20,],col.symbols="green",pch=3)
 par(triax.return$oldpar)
\end{ExampleCode}
\end{Examples}

