\HeaderA{dispersion}{Display a measure of dispersion}{dispersion}
\aliasA{dispbars}{dispersion}{dispbars}
\keyword{misc}{dispersion}
\begin{Description}\relax
Display line/cap bars at specified points on a plot representing
measures of dispersion.
\end{Description}
\begin{Usage}
\begin{verbatim}
 dispersion(x,y,ulim,llim=ulim,intervals=TRUE,arrow.cap=0.01,arrow.gap=NA,
  type="a",fill=NA,...)
 dispbars(x,y,ulim,llim=ulim,intervals=TRUE,arrow.cap=0.01,arrow.gap=NA,...)
\end{verbatim}
\end{Usage}
\begin{Arguments}
\begin{ldescription}
\item[\code{x,y}] x and y position of the centers of the bars
\item[\code{ulim,llim}] The extent of the dispersion measures.
\item[\code{arrow.cap}] The width of the cap at the outer end of each bar
as a proportion of the width of the plot.
\item[\code{arrow.gap}] The gap to leave at the inner end of each bar.
Defaults to two thirds of the height of a capital "O".
\item[\code{intervals}] Whether the limits are intervals (TRUE) or absolute values
(FALSE).
\item[\code{type}] What type of display to use.
\item[\code{fill}] Color to fill between the lines if \samp{type} is not NULL and
\samp{fill} is not NA.
\item[\code{...}] additional arguments passed to \samp{arrows} or \samp{lines}
depending upon \samp{type}.
\end{ldescription}
\end{Arguments}
\begin{Details}\relax
\samp{dispersion} displays a measure of dispersion on an existing plot.
Currently it will display either vertical lines with caps (error bars) or lines
that form a "confidence band" around a line of central tendency. If \samp{fill}
is not NA and \samp{type} is \samp{"l"}, a polygon will be drawn between the
confidence lines. Remember that any points or lines within the confidence band
will be obscured, so these will have to be redrawn.

The \samp{intervals} argument allows the user to pass the limits as either intervals (the default) or absolute values.

\samp{dispbars} is now a call to \samp{dispersion} with \samp{type} set to
\samp{a}.

If \samp{arrow.gap} is greater than or equal to the upper or lower
limit for a bar, \samp{segments} is used to draw the upper and
lower caps with no bars to avoid zero length arrows.
\end{Details}
\begin{Value}
nil
\end{Value}
\begin{Author}\relax
Jim Lemon
\end{Author}
\begin{SeeAlso}\relax
\samp{\LinkA{arrows}{arrows}}, \samp{\LinkA{segments}{segments}},\samp{\LinkA{lines}{lines}}
\end{SeeAlso}
\begin{Examples}
\begin{ExampleCode}
 disptest<-matrix(rnorm(200),nrow=20)
 disptest.means<-rowMeans(disptest)
 row.order<-order(disptest.means)
 se.disptest<-unlist(apply(disptest,1,std.error))
 plot(disptest.means[row.order],main="Dispersion as error bars",
  ylim=c(min(disptest.means-se.disptest),max(disptest.means+se.disptest)),
  xlab="Occasion",ylab="Value")
 dispersion(1:20,disptest.means[row.order],se.disptest[row.order])
 plot(disptest.means[row.order],main="Dispersion as confidence band",
  ylim=c(min(disptest.means-se.disptest),max(disptest.means+se.disptest)),
  xlab="Occasion",ylab="Value")
 dispersion(1:20,disptest.means[row.order],se.disptest[row.order],type="l",
  fill="#eeccee",lty=2)
 # remember to redraw the points
 points(disptest.means[row.order])
\end{ExampleCode}
\end{Examples}

