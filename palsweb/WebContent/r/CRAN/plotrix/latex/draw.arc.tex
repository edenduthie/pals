\HeaderA{draw.arc}{Draw arc}{draw.arc}
\keyword{misc}{draw.arc}
\begin{Description}\relax
Draw one or more arcs using classic graphics.
\end{Description}
\begin{Usage}
\begin{verbatim}
 draw.arc(x=1,y=NULL,radius=1,angle1=deg1*pi/180,angle2=deg2*pi/180, 
  deg1=0,deg2=45,n=35,col=1,...)
\end{verbatim}
\end{Usage}
\begin{Arguments}
\begin{ldescription}
\item[\code{x}] x coordinate of center.  Scalar or vector. 
\item[\code{y}] y coordinate of center.  Scalar or vector. 
\item[\code{radius}] radius.  Scalar or vector.  
\item[\code{angle1}] Starting angle in radians. Scalar or vector. 
\item[\code{angle2}] Ending angle in radians. Scalar or vector. 
\item[\code{deg1}] Starting angle in degrees. Scalar or vector. 
\item[\code{deg2}] Ending angle in degrees. Scalar or vector. 
\item[\code{n}] Number of polygons to use to approximate the arc. 
\item[\code{col}] Arc colors. 
\item[\code{...}] Other arguments passed to segments.  Vectorization 
is not supported for these. 
\end{ldescription}
\end{Arguments}
\begin{Details}\relax
Draws one or more arcs from \code{angle1} to \code{angle2}.
If \code{angle1} is numerically greater than \code{angle2},
then the angles are swapped.

Be sure to use an aspect ratio of 1 as shown in the example to avoid
distortion.
\end{Details}
\begin{Value}
Returns a matrix of expanded arguments invisibly.
\end{Value}
\begin{Author}\relax
Gabor Grothendieck
\end{Author}
\begin{Examples}
\begin{ExampleCode}

   plot(1:10, asp = 1,main="Test draw.arc")
   draw.arc(5, 5, 1:10/10, deg2 = 1:10*10, col = "blue")
   draw.arc(8, 8, 1:10/10, deg2 = 1:10*10, col = 1:10)

   # example taken from post by Hans Borcher:
   # https://stat.ethz.ch/pipermail/r-help/2009-July/205728.html
   # Note setting of aspect ratio to 1 first.
   curve(sin(x), 0, pi, col="blue", asp=1)
   draw.arc(pi/2, 0, 1, deg1=45, deg2=135, col="red")

\end{ExampleCode}
\end{Examples}

